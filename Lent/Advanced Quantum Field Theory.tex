\documentclass{article}
\usepackage[utf8]{inputenc}
\usepackage{mathtools}
\usepackage{amssymb}
\usepackage{graphicx}
\usepackage{listings}
\usepackage{float}
\usepackage{gensymb}
\usepackage{amsthm}
\usepackage{longtable}
\usepackage{adjustbox}
\usepackage{physics}
\usepackage{dsfont}
\usepackage{cancel}

\theoremstyle{definition}
\newtheorem{definition}{Definition}
\newtheorem{example}{Example}
\newtheorem{theorem}{Theorem}
\newtheorem{claim}{Claim}

\title{Quantum Information Theory}
\author{quinten tupker}
\date{January 22 2021 - \today}

\begin{document}

\maketitle

\section*{Introduction}

These notes are based on the course lectured by Professor Matthew Wingate in
Lent 2020. 
This was lectured online due to measures taken to counter the spread of Covid-19
in the UK. These are not necessarily an accurate representation of what was
lectures, and represent solely my personal notes on the content of the course,
combinged with probably, very very many personal notes and digressions... Of
course, any corrections/comments would be appreciated.

[the lecturer outlines the course] This course is an extension of the Michaelmas
Quantum Field Theory course that introduces renormalisation and the path
integral formulation of quantum field theory.

\section*{The Path Integral in Quantum Mechanics}

We start by reformulating the Schr\"{o}dinger equation as an integral equation,
which turns out to be a path integral. Anyways, starting with Schr\"{o}dinger's
equation for a Hamiltonian $H(x, p), [x, p] = i\hbar$ with

\begin{equation}
  H = \frac{p^2}{2m} + V(x)
\end{equation}

we have

\begin{equation}
  i\hbar \partial_t \ket{\psi(t)} = H \ket{\psi(t)} \implies \ket{\psi(t)} = e^{-i H t / \hbar}
  \ket{\psi(0)}
\end{equation}

where in the Schr\"{o}dinger picture the states evolve, but the operators remain
constant, and the wavefunction $\Psi(x, t) = \bra{x} \ket{\psi(t)}$. As such we
can rewrite our equation as

\begin{equation}
  \bra{x} H \ket{\psi(x)} = \left( \frac{-\hbar^2}{2m} \partial_x^2 + V(x) \right) \bra{x} \ket{\psi(t)}
\end{equation}

so we can write

\begin{align*}
  \Psi(x, t) &= \bra{x} \ket{\psi(t)} \\
             &= \bra{x} e^{-i H t / \hbar} \ket{\psi(0)} \\
             &= \int_{-\infty}^\infty dx_0 \bra{x} e^{-i H t / \hbar} \ket{x_0} \bra{x_0} \ket{\psi(0)} \\
             &= \int_{-\infty}^\infty dx_0 K(x, x_0, t) \Psi(x_0, 0)
\end{align*}

for \textbf{kernel} $K(x, x_0, t) = \bra{x} e^{-i H t / \hbar} \ket{x_0}$. Now,
if it is hard to calculate $K$ for large $t$, it can be beneficial to split this
into many intervals for many values of $t$, such as $0 = t_0 < t_1 < \dots < t_n
< t_{n + 1} = T$ leaving

\begin{equation}
  K(x, x_0, T) = \int_{-\infty}^\infty \prod_{r = 1}^n dx_r \bra{x_{r + 1}} e^{- iH(t_{r + 1} - t_r) / \hbar} \ket{x_r} \bra{x_1} e^{-iH (t_1 - t_0) / \hbar} \ket{0}
\end{equation}

which is in a sense an integral over all possible sequences of values of $x$. 

In free field theory ($V = 0$) this can be explicitly evaluated using a Gaussian
integral by rewriting things in the momentum basis as (use $\bra{x} \ket{p} =
e^{i px / \hbar}$)

\begin{align*}
  K_0(x, x', t) &= \bra{x} e^{\frac{-i p^2 t}{2m \hbar}} \int \frac{dp}{2\pi \hbar} \ket{p} \bra{p} \ket{x'} \\
                &= \int_{-\infty}^\infty \frac{dp}{2 \pi \hbar} e^{\frac{-ip^2 t}{2m \hbar}} e^{ip (x - x') / \hbar}\\
                &= e^{\frac{ip(x - x')^2}{2\hbar t}} \sqrt{\frac{m}{2\pi i \hbar t}}
\end{align*}

where we note that the limit as $t \to 0$ is $\delta(x - x')$ which indeed
matches $\bra{x} \ket{x'} = \delta(x - x')$ as expected.

Now in an interacting theory, we struggle with the Baker-Campbell-Hausdorff fact
that $e^A e^B \neq e^{A + B}$ so using Suzuki-Trotter we separate into steps
size $t_{r + 1} - t_r = \delta t << T$ meaning that

\begin{equation}
  e^{-iH \delta t / \hbar} \approx e^{\frac{-ip^2 \delta t}{2m \hbar}} e^{\frac{-i V(x) \delta}{\hbar}} (1 + O(\delta t^2))
\end{equation}

so for $T = n \delta t$ we find that

\begin{equation}
  K(x, x_0, T) = \int \prod_{r = 1}^n dx_r \left( \frac{m}{2\pi i \hbar \delta t} \right)^{\frac{n + 1}{2}}
  e^{i \sum_{r = 0}^n \left( \frac{m}{2 \hbar} \left( \frac{x_{r + 1} - x_r}{\delta t} \right)^2
      - V(x_r) / \hbar \right) \delta t}
\end{equation}

which in the limit $n \to \infty, \delta t \to 0$ while keeping $T$ constant
leaves

\begin{equation}
  \frac{1}{\hbar} \int_0^T dt \left( \frac{1}{2}m \dot{x}^2 - V(x) \right) = \int_0^T dt L(x, \dot{x}) = S
\end{equation}

for classical Lagrangian $L$ and action $S$. This is what we refer to as a path
integral or function integral:

\begin{equation}
  K(x, x_0, t) = \int \mathcal{D}x e^{i S / \hbar}
\end{equation}

where $\mathcal{D} x$ is the limit describd above. Of course, many questions
about the existence and uniqueness, etc. of such limits exists, and in fact
often this limit does not exist, but in the cases we are interested it, it works
well enough... [End of lecture 1]

We make the following remarks

\begin{itemize}
\item In the classical limit $\hbar \to 0$ the lowest frequencies
  dominate $K$. This is equivalent to Hamilton's principle (the principle of
  least action), as expected.
\item it is common and helpful to extend analytically to imaginary time $\tau =
  it$ leaving $\bra{x} e^{-H\tau / \hbar} \ket{x_0} = \int \mathcal{D}x e^{-S /
    \hbar}$ which has better convergence properties and is easier to interpret
  than the complex version (Hamilton's principle appears more easily as well). 
\end{itemize}

\section{Integrals and their diagrammatic expansion}

The above considered quantum mechanics, which is in a sense the $0 + 1$
dimension vrsion of QFT (since $x$ is treated as an operator, while $t$ is
treated as a variable). To move to more general QFT, we start, strangely, with 0
dimensional QFT, for $\phi : \{ \cdot \} \to \mathbb{R}$ a field on a single
point. Here,

\begin{equation}
  \mathcal{Z} = \int_\mathbb{R} d\phi e^{-S(\phi) / \hbar}
\end{equation}

where we assume $S$ is an even polynomial in $\phi$ for convergence reasons, and
we are interested in expectation values

\begin{equation}
  \langle f \rangle = \frac{1}{\mathcal{Z}} \int d\phi f(\phi) e^{-S(\phi) / \hbar}
\end{equation}

\subsection{Free Theory}

For $N$ fields $\phi_a, a = 1, \dots, N$, let $S(\phi) = \frac{1}{2} \phi^T m
\phi$ for a symmetric positive definite matrix $m = P\Delta P^T$ for orthogonal
$P$. As such, we can write this essentially Gaussian integral as

\begin{equation}
  \mathcal{Z}_0 = \int d^N \phi e^{-\frac{1}{2 \hbar} \phi^T m \phi} = \sqrt{\frac{(2\pi\hbar)^N}{\det m}}
\end{equation}

From here, we can turn this into a generating function to calculate expectation
values by taking derivatives by turning $S_0(\phi) \mapsto S_0(\phi) - J^T
\phi$, and writing $\mathcal{Z}_0 = \mathcal{Z}_0(J)$ (now a generating
function(al) - functional later on). We then remark

\begin{equation}
  \mathcal{Z}_0(J) = \mathcal{Z}_0(0) e^{-\frac{1}{2\hbar} J^T m^{-1} J}
\end{equation}

Then we can calculate the correlation functions as

\begin{equation}
  \langle \phi_a \phi_b \rangle = \frac{1}{Z_0(0)} \hbar^2 \partial_{J_a} \partial_{J_b} \mathcal{Z}_0(J) |_{J = 0}
  = h(m^{-1})_{ab}
\end{equation}

Conveniently, this can be diagrammatically interpreted as a connecting two
vertices on indices $a, b$ with an undirected edge. We can generalise this to
linear operator $l(\phi) = \sum l_a \phi_a$ as (for $p$ such operators)

\begin{equation}
  \langle l^{(1)}(\phi) \dots l^{(p)}(\phi) \rangle = \hbar^p \prod_{i = 1}^p l^{(i)}(\partial_J) e^{\frac{1}{\hbar} J^T m^1 J}
\end{equation}

If $p$ is odd, this is always 0 by symmetry, but if $p$ is even this corresponds
to a linear combination of products $m_{ab}^{-1} m_{cd}^{-1} \dots$

\begin{example}
  For $p=4, l^{(1)}_a = \delta_{ab}, l^{(2)}_a = \delta_{ac}, l^{(3)}_a =
  \delta_{ad}, l^{(4)}_a = \delta_{af}$ then
  \begin{equation}
    \langle \phi_b \phi_c \phi_d \phi_f \rangle = \hbar^2 (m_{bc}^{-1} m_{df}^{-1} + m_{bd}^{-1} m_{cf}^{-1} +
    m_{bf}^{-1} m_{cd}^{-1})
  \end{equation}
  which also corresponds to the ways in which we can connect four vertices with
  undirected edges
  \begin{figure}[H]
    \centering
    \includegraphics[width=7cm]{res/AQFT/connecting_four_vertices}
    \label{connecting_four_vertices}
  \end{figure}
\end{example}

[End of lecture 2]

\subsection{Interacting Theory}

We start investigating interacting theory by doing a series expansion of

\begin{equation}
  \int_{\mathbb{R}^N} \d^N \phi f(\phi) e^{-S / \hbar}
\end{equation}

in $\hbar$. However, we find that in general, the radius of convergence of these
perturbed series is 0, since if we take $\hbar < 0$ these do not converge. As
such, we get asymptotic behaviour along the lines described by saying that

\begin{definition}
  $I(\hbar)$ is \textbf{asymptotic} to $\sum_{n = 0}^\infty c_n \hbar^n$
  (denoted by $\sim$) if $\lim_{\hbar \to 0^+} \frac{1}{\hbar^N} | I(\hbar) -
  \sum_{n = 0}^N c_n \hbar^n | = 0$ for fixed $N$.
\end{definition}

This is much weaker than convergence since we find that adding new terms may in
fact make things worse. But it does allow us to account for transcendental terms
like $e^{-1 / \hbar^2} ~ 0$ (called \textbf{nonperturbative contributions}).

Now we work out the case where

\begin{equation}
S(\phi) = \frac{1}{2} m^2 \phi^2 + \frac{\lambda}{4!} \phi^4
\end{equation}

and expand about the minimum of $S$ at $\phi=0$ to get

\begin{equation}
\mathcal{Z} = \int d\phi e^{-S / \hbar} = \int d\phi e^{-S / \hbar} 
\sum_v^\infty \frac{1}{v!} \left( \frac{-\lambda}{4! \hbar} \phi^4 \right)^v
\end{equation}

but since we don't converge, we certainly can't swap $\int$ and $\sum$ so here
we first truncate, and then swap. As such, if we write $x = \frac{1}{2\hbar} m^2
\phi^2$,

\begin{equation}
\mathcal{Z} \sim \frac{\sqrt{2\hbar}}{m} \sum_{v = 0}^N \frac{1}{v!} \left( 
\frac{-\hbar \lambda}{4! m^4} \right)^v 2^{2v} \int_0^\infty dx e^x x^{2v
+ 1/2 - 1}
\end{equation}

where the integral is just the gamma function $\Gamma(2v + 1/2) = \frac{(4v)!
  \sqrt{\pi}}{4^{2v} (2v)!}$ so

\begin{equation}
\mathcal{Z} \sim \frac{\sqrt{2\hbar}}{m} \sum_{v = 0}^N \frac{1}{v!} \left( 
\frac{-\hbar \lambda}{4! m^4} \right)^v \frac{1}{(4!) v} \frac{(4v)! \sqrt{\pi}}{2^{2v} (2v)!}
\end{equation}

(the lecture notes seem to omit the $\sqrt{\pi}$). Here the $\frac{1}{(4!) v}$
comes from expanding the interaction term $e^{-S_1 / \hbar}$ and the second
fraction comes from the number of ways to pair $4v$ fields with $v$ copies of
$\phi^4$. Applying stirling's formula, this series grows as $v!$ so this series
is definitely not convergent, but still asymptotic. Now, as before, we want to
insert our $J$ somehow to get a generating function. Here we get

\begin{align*}
\mathcal{Z}(J) &= \int d\phi e^{-\frac{1}{\hbar} (S_0(\phi) + S_1(\phi) - J\phi)} \\
&= e^{-\frac{1}{\hbar} S_1(\hbar \partial_J)} \int d\phi e^{-\frac{1}{\hbar} 
(S_0(\phi) - J\phi)} \\
&\sim e^{-\frac{\lambda}{4! \hbar} (\hbar \partial_J)^4} e^{\frac{1}{2\hbar} J^T m^{-1} J}
\end{align*}

which works out to

\begin{equation}
  \mathcal{Z}(J) \sim \sum_{v=0}^N \frac{1}{v!} \left( \frac{-\lambda}{4! \hbar}
    (\hbar \partial_J)^4 \right)^v \sum_{p = 0} \frac{1}{p!} \left( \frac{1}{2\hbar}
    J^T m^{-1} J \right)^p
\end{equation}

where diagramatically $v$ corresponds to the number of vertices, and $p$ to the
number of propagators.

\begin{figure}[H]
  \centering
  \includegraphics[width=7cm]{res/AQFT/diagram_interpretation}
  \label{diagram_interpretation}
\end{figure}
 
where in order to get a nonzero term we need the number of derivatives ($4v$
vertex line ends - 4 per vertex here) to match the number of sources ($2p$ for
each propagator) to match. However, we can have a predetermined number of
external sources $E = 2p - 4v$. For example, the first two nontrivial terms in
the $Z(0)$ expansion for $E = 0$ are $(v, p) = (1, 2), (2, 4)$ which corresponds
diagrammatically to

\begin{figure}[H]
  \centering
  \includegraphics[width=7cm]{res/AQFT/quartic_series_diagram}
  \label{quartic_series_diagram}
\end{figure}

Note that each diagram may have a factor in front of it determined by how often
it repeats itself (affect by the product rule in taking derivatives). [end of
lecture 3]

To work out these prefactors, consider the first non-constant term above, the
figure eight. We can split this into its components: a vertex and two
propagators - it's so-called ``pre-diagram.'' We can work out the prefactor by
considering the number of ways to connect these while still forming a figure
eight, $A = 4!$, and dividing by a denominator given by the coefficients in the
series

\begin{equation}
  F = v! (4!)^v (p!) 2^p = 1 \cdot 4! \cdot 2 \cdot 2^2 = 4! 2^3
\end{equation}

leaving prefactor $A / F = 1 / 8$. Note here that $F$ accounts for

\begin{itemize}
\item $v!$ ways to permute the vertices
\item $4!$ ways to permute the vertex legs
\item $p!$ ways to permute the propagator legs
\item $2^p$ ways to swap propagator direction
\end{itemize}

Now, another interpretation is that $A / F = 1 / S$ where $S$ is the
\textbf{symmetry factor} counting the number of ways of redrawing the unlabelled
graph to leave its overall structure the same (the number of graph
isomorphisms). So, for example, for the figure eight, we get $S = 2 \times 2
\times 2 = 8$ for swapping the direction of loop 1, swapping the direction of
loop 2, and swapping loops 1 and 2. Similarly for the basketball, we get $S = 4!
\cdot 2 = 48$ for $4!$ ways to rearrange the lines and $2$ ways to swap
vertices (what happened to swapping the orientations of the lines?). Working by
brute force, to verify we get $A = 8 \cdot 6 \cdot 4 \cdot 2 \cdot 4! = 3^2
2^{10}$ for the number of ways to connect propagators and the number of their
permutations. Similarly, $F = 2(4!)^24! 2^4 = 3^3 2^{14},$ leaving $A / F = 1 /
48 = 1 / S$. Overall, in this case we get

\begin{equation}
  \mathcal{Z}(0) / \mathcal{Z}_0(0) = 1 - \frac{\hbar \lambda}{8m^4} +
  \frac{\hbar^2 \lambda^2}{m^8} \left( \frac{1}{48} + \frac{1}{16} +
    \frac{1}{128} \right) + O(\hbar^3)
  = 1 - \frac{\hbar \lambda}{8 m^4} + \frac{35}{384} \frac{\hbar^2 \lambda^2}
  {m^8} + \dots
\end{equation}

Now to work out the $E = 2$ case we get that

\begin{figure}
  \centering
  \includegraphics[width=7cm]{res/AQFT/E_2_expansion}
  \label{E_2_expansion}
\end{figure}

where we get disconnected graphs of a different kind, and also get loose outputs
really. Importantly, we can factor out the ``vacuum bubbles'' (or the $E = 0$
case described earlier)

\begin{figure}
  \centering
  \includegraphics[width=7cm]{res/AQFT/vacuum_bubble_product}
  \label{vacuum_bubble_product}
\end{figure}

Note that this corresponds to the expectation value of $\langle \phi \phi
\rangle$

\begin{figure}
  \centering
  \includegraphics[width=7cm]{res/AQFT/E_2_mean}
  \label{E_2_mean}
\end{figure}

\subsection{Effective Actions}

Our next step is to simplify these calculations by showing that we only have to
work hard on connected graphs. In particular, we define \textbf{effective
  action} $W(J) = -\hbar \ln Z(J)$
and a diagram $D$. Any such $D$ can be written as a product of connect diagrams
as

\begin{equation}
  D = \frac{1}{S_D} \prod_i (C_i)^{n_i}
\end{equation}

where each $C_i$ is a distinct diagram, and we assume each $C_i$ contains its
own symmetry factor, meaning that $S_D = \prod_i n_i!$ is only the number of
ways to rearrange the various connected diagrams. Consequently,

\begin{align*}
  \mathcal{Z} / \mathcal{Z}_0
  &= \sum_{\{n_i\}} \prod_i \frac{1}{n_i!} (C_i)^{n_i} \\
  &= \prod_{i=1}^\infty \sum_{n_i} \frac{1}{n_i!} (C_i)^{n_i} \\
  &= e^{\sum_i C_i} \\
  &= e^{-(W - W_0) / \hbar}
\end{align*}

leaving

\begin{equation}
  \mathcal{Z} / \mathcal{Z}_0 = e^{\sum_i C_i} = e^{-(W - W_0) / \hbar}
\end{equation}

which is quite a remarkable decomposition into only connected graphs. [End of
lecture 4] Let's work out how to use the effective action, $W$ in general.

\begin{example}
  Consider 0-dimension action with two fields
  \begin{equation}
    S(\phi, \chi) = \frac{m^2}{2} \phi^2 + \frac{M^2}{2} \chi^2 +
    \frac{\lambda}{4} \phi^2 \chi^2
  \end{equation}
  which has Feynman rules
  \begin{figure}[H]
    \centering
    \includegraphics[width=8cm]{res/AQFT/lec_5_effective_action_feynman_rules}
    \caption{Feynman Rules}
    \label{lec_5_effective_action_feynman_rules}
  \end{figure}
  and so we have a sum of connected diagrams given by
  \begin{figure}[H]
    \centering
    \includegraphics[width=8cm]{res/AQFT/lec_5_connected_diagrams}
    \caption{Connected Diagrams Expansion}
    \label{lec_5_connected_diagrams}
  \end{figure}
  in the so-called ``full theory''. Note here that the free theory involves
  \begin{figure}[H]
    \centering
    \includegraphics[width=8cm]{res/AQFT/lec_5_free_theory}
    \caption{Free Theory Expected Value}
    \label{lec_5_free_theory}
  \end{figure}

  We can reduce the complexity of these calculations by removing the explicit
  $\chi$ depending by ``integrating it out'' (which can make sense if $\chi$ is
  very massive, for example, so doesn't contribute strongly), then we define our
  effective action to be $W(\phi)$ such that
  \begin{equation}
    e^{-W(\phi) / \hbar} = \int d\chi e^{-S(\phi, \chi) / \hbar}
  \end{equation}
  where we in effect treat $\phi^2 \chi^2$ as a local source for $\chi^2$ (with
  $J = - \phi^2 \lambda / 4$). Consequently, correlatino functions are given by
  \begin{equation}
    \langle f(\phi) \rangle
    = \frac{1}{\mathcal{Z}} \int d \phi d \chi f(\phi) e^{-S(\phi, \chi) / \hbar}
    = \frac{1}{\mathcal{Z}} \int d \phi f(\phi) e^{-W(\phi) / \hbar}
  \end{equation}
  In this special case, we can evaluate the integral explicitly as
  \begin{equation}
    \int d\chi e^{-S(\phi, \chi) / \hbar} = e^{-m^2 \phi^2 / 2 \hbar}
    \sqrt{\frac{2 \pi \hbar}{M^2 + \lambda \phi^2 / 2}}
  \end{equation}
  meaning
  \begin{equation}
    W(\phi) = \frac{1}{2} m \phi^2 + \frac{\hbar}{2}
    \ln(1 + \frac{\lambda}{2M^2} \phi^2) + \frac{\hbar^2}{2}
    \ln \left( \frac{M^2}{2 \pi \hbar} \right)
  \end{equation}
  Here the last constant term does not effect QFT, and is ignored. It does,
  however, have interpretations relating to the energy density of the universe,
  and thus, the cosmological constant. Expanding in $\phi$ (since $\phi = 0$ is
  the local minimum here) gives
  \begin{equation}
    W(\phi) = \frac{m_{eff}^2 \phi^2}{2} + \frac{\lambda_4}{4!} \phi^4 + \dots +
    \frac{\lambda_{2k}{(2k)!}} \phi^{2k}
  \end{equation}
  where $m_{eff}^2 = m^2 + \frac{\hbar \lambda}{4M^2}, \lambda_{2k} = (-1)^{k +
    1} \hbar \frac{(2k)!}{2^{k + 1}k} \frac{\lambda^k}{M^{2k}}$. (notice how we
  get many more terms in the effective theory than the full theory. This is a
  standard effect.). However, integration, as done here, is usually not
  possible, so we resort to pertubations, treating $
  \frac{\lambda}{4} \phi^2 \chi^2$ as a source with rules
  \begin{figure}[H]
    \centering
    \includegraphics[width=8cm]{res/AQFT/lec_5_effective_source}
    \caption{$\frac{\lambda}{4} \phi^2 \chi^2$ as a source}
    \label{lec_5_effective_source}
  \end{figure}
  leaving
  \begin{figure}[H]
    \centering
    \includegraphics[width=8cm]{res/AQFT/lec_5_effective_pertubation}
    \caption{Effective Action Expansion}
    \label{lec_5_effective_pertubation}
  \end{figure}
  which as usual can be used to calculate correlation functions
  \begin{figure}[H]
    \centering
    \includegraphics[width=8cm]{res/AQFT/lec_5_effective_correlation}
    \caption{Effective Action Correlation}
    \label{lec_5_effective_correlation}
  \end{figure}
  which is the same as the full theory.
\end{example}

\subsection{Quantum Effective Action $\Gamma$}

The previous effective action accounts for some quantum effects, we can also
introduce the quantum effective action, which accounts for all quantum effects
(?). As such we first define
\begin{equation}
  \Phi = -\partial_J W = \langle \phi \rangle
\end{equation}
for $J \neq 0$. Then we can do a Legendre transform (we assume convexity, and in
practice, this is justified)
\begin{equation}
  \Gamma(\Phi) = W(J) + \Phi J
\end{equation}
which has the property that $\partial_\Phi \Gamma = J$ (so at $J = 0$ we get
$0$, meaning we have a local extremum). In higher dimensions, we can use this to
define the \textbf{effective/quantum} potential $V(\Phi)$ which can be more useful than
the action
\begin{equation}
  \Gamma(\Phi) = \int d^4x \left( -V(\Phi)
    - \frac{1}{2} \partial^\mu \Phi \partial_\mu \Phi + \dots \right)
\end{equation}

We can draw analogies to statistical mechanics with this Legendre transform, and
the Gibbs free energy, for example. [End of lecture 5]

We can interpret the effective action $\Gamma(\Phi)$ nicely in Feynman diagrams,
but first

\begin{definition}
  we define a \textbf{bridge} to be an internal line of a connected graph such
  that if it is cut, the graph becomes disconnected.
  (in the context of Feynman diagrams, graphs can contain both internal and
  external lines.)
\end{definition}

\begin{definition}
  Furthermore, we call a connected graph a \textbf{one particle irreducible} or
  1PI if it contains no bridges.
\end{definition}

\begin{claim}
  From here we claim that when we expand $\Gamma(\Phi)$ we sum over the 1PI
  graphs of the $S(\phi)$ theory with vertices
  \begin{equation}
    \gamma(\Phi) = \Gamma^{(0)} + \Gamma^{(1)} \Phi +
    \frac{1}{2!} \Gamma^{(2)} \Phi^2 + \dots
  \end{equation}
  which can be thought of interactions.
\end{claim}

\begin{example}
  For example, in a theory with $\phi^3, \phi^4$ terms we may have
  \begin{figure}[H]
    \centering
    \includegraphics[width=7cm]{res/AQFT/lec_6_effective_action_expansion}
    \label{lec_6_effective_action_expansion}
    \caption{Expanding the Effective Action}
  \end{figure}
  where the $3, 4$ powers come from the number of vertices connecting to each
  1PI graph.
\end{example}

But we should provide some justification for this statement. Let's imagine we
treat $\Gamma(\Phi)$ just as we treat the normal action, and so we get

\begin{equation}
  e^{W_\Gamma(J) / g} = \int d\Phi e^{-(\Gamma(\Phi) - J \Phi) / g}
\end{equation}

for fictitious Planck's constant $g$ and sum of connected diagrams $W_\Gamma$.
We can expand this as $W_\Gamma(J) = \sum_{l = 0}^\infty g^l W_\Gamma^{(l)} (J)$
and then, as with $S(\phi), W(J)$ only the tree graphs (so the graphs without
loops) contribute to $W_\Gamma(J)$ as $g \to 0$.

\begin{equation}
  \lim_{g \to 0} W_\Gamma (J) = W_\Gamma^{(0)}(J)
\end{equation}

Note also that as $g \to 0$ the minimum dominates, so we have $\partial_\Phi
\Gamma = J$ (Legendre transform) and so

\begin{equation}
  W_\Gamma(J) = W_J^{(0)}(J) = \Gamma(\Phi) - J\Phi = W(J)
\end{equation}

where $W(J)$ is given by

\begin{equation}
  e^{-W(J) / \hbar} = \int d\phi e^{-(S(\phi) - J\phi) / \hbar}
\end{equation}

meaning that the sum of connected diagrams from $W(J)$ can be represented as a
sum of tree diagrams in $W_\Gamma(J)$ for $g \to 0$ (is this only true for
$\hbar \to 0$?).

Note also, that for a theory with $N$ fields $\phi_a$ we have

\begin{align*}
  \langle \phi_a \phi_b \rangle_J^{conn}
  &= \langle \phi_a \phi_b \rangle_J - \langle \phi_a \rangle_J
    \langle \phi_b \rangle_J \\
  &= -\hbar \partial_{J_a} \partial_{J_b} W \\
  &= \hbar \partial_{J_a} \Phi_b \\
  &= \hbar \left( \partial_{\Phi_b) J_a} \right) \\
  &= \hbar (\partial_{\Phi_a} \partial_{\Phi_b} \Gamma)^{-1}
\end{align*}

so the full propagator including loops is $\hbar$ times the inverse quadratic
term in $\Gamma$.

\subsection{Fermions}

The current theory works well for bosons, but to allow for the antisymmetric
nature of fermions, we need a little extra work. Here we introduce the
\textbf{Grassmannian numbers} $\theta_a$ obeying
\begin{equation}
  \theta_a \theta_b = - \theta_b \theta_a
\end{equation}
meaning for example that $\theta_a^2 = 0$ always. This last property is
convenient, since it means in particular, that all series expansions are finite.
Also, for $\phi \in \mathbb{C}$ we simply define $\phi \theta = \theta \phi$.
Then we have that series here take the form
\begin{equation}
  F(\theta) = f + \phi_a \theta_a + \frac{1}{2!} g_{ab} \phi_a \phi_b + \dots +
  \frac{1}{N!} k_{a_1 \dots a_N} \theta_{a_1} \dots \theta_{a_N}
\end{equation}
which is indeed a finite sum. Notice also that we require here that the tensors
of coefficients are totally antisymmetric in the swapping of indices $g_{ab} =
-g_{ba}$. Also note that even products of Grassmannians are not Grassmannian, so
$\theta_a \theta_b$ is not Grassmannian since
\begin{equation}
  (\theta_1 \theta_2) (\theta_3 \theta_4) = (\theta_3 \theta_4) (\theta_1 \theta_2)
\end{equation}

Now, we extend to some extra definitions for other operations. Here we define
\textbf{Grassmannian differentiation} such that
\begin{equation}
  \partial_{\theta_a} (\theta_b F(\theta)) = - \theta_b \partial_{\theta_a} F +
  \delta_{ab} F
\end{equation}

Integration is a bit different, but we require linearity of differentiation, and
so if $\eta$ is a Grassmannian constant we have that $\int d\theta \eta = 0$, so
we see that $\int d\theta = 0, \int d \theta \theta = 1$ (normalising). As such
we define \textbf{Grassmannian integration} such that
\begin{equation}
  \int d\theta (f + \phi \theta) =
\end{equation}
These rules are called the \textbf{Berezin rules}. We also notice that $\int
d\theta \partial_\theta F(\theta) = 0$. [End of lecture 6]

From here we note that if we take an integral $\int d^N\theta F$ the only term
that does not vanish is the one that has each $\theta_a$ appearing exactly once.
In particular, we find that
\begin{equation}
  \int d\theta^N \theta_{a_1} \dots \theta_{a_N} = \epsilon^{a_1 \dots a_N}
\end{equation}

Now suppose we want to change variables $\theta_a' = X_{ab}\theta_b, X_{ab} \in
\mathbb{C}$ then

\begin{align*}
  \int d^N \theta \theta_{a_1}' \dots \theta_{a_N}'
  &= X_{a_1b_1} \dots X_{a_Nb_N} \int d^N \theta \theta_{b_1} \dots \theta_{b_N} \\
  &= X_{a_1b_1} \dots X_{a_Nb_N} \epsilon^{b_1 \dots b_N} \\
  &= \det (X) \epsilon^{a_1 \dots a_N} \\
  &= \det (X) \int d^N\theta' \theta_{a_1}' \dots \theta_{a_N}'
\end{align*}
which leads to an interesting contrast with commuting scalars:
\begin{align}
  \theta' = X\theta \implies d^N \theta' = \frac{1}{\det(X)} d^N \theta \\
  \phi' = Y \phi \implies d^N \phi' = \det(Y) d^N\phi
\end{align}

\subsubsection{Free fermion field theory (0 dimensions)}

For $N = 2m$ fermionic fields $m \in \mathbb{N}$ we see that the action must
take the form $S = \frac{1}{2} A_{ab} \theta_a \theta_b$ since $S$ must be
bosonic. Then we see that
\begin{align*}
  \mathcal{Z}_0
  &= \int d^{2m} \theta e^{-S / \hbar} \\
  &= \frac{(-1)^m}{(2\hbar)^m m!} \int d^{2m}\theta A_{a_1a_2} \dots
    A_{a_{2m - 1}a_{2m}} \theta_{a_1} \dots \theta_{a_{2m}} \\
  &= \frac{(-1)^m}{(2\hbar)^m m!} \epsilon^{a_1 \dots a_{2m}}
    A_{a_1a_2} \dots A_{a_{2m - 1}a_{2m}} \\
  &= \frac{(-1)^m}{\hbar^m} \text{Pf}(A) = \pm \sqrt{\frac{\det(A)}{\hbar^N}}
\end{align*}
where we remove the exponential by noticing that in its power series the only
term that does not vanish is the term that contains all $\theta$s exactly once.
We also note that the Pfaffian of an antisymmetric matrix here can be defined as
\begin{equation}
  \text{Pf}(A) = \frac{1}{2^m m!} \epsilon^{a_1 \dots a_{m}}
  A_{a_1a_2} \dots A_{a_{2m - 1}a_{2m}}
\end{equation}
(the Pfaffian is necessary content in this course. It apparently also has uses
in supersymmetry.) Consequently, we can summarise our results as
\begin{align}
  \mathcal{Z}_{0, bosons} = \sqrt{\frac{(2\pi \hbar)^N}{\det(M)}} \\
  \mathcal{Z}_{0, fermions} = \pm \sqrt{\frac{\det(A)}{\hbar^N}}
\end{align}
As such in computer simulations, which struggle with Grassmanns, one often
treats fermions as bosons with $M = A^{-1}$...

\subsubsection{Adding external sources}

To add external source we introduce Grassmann $\eta$ and say that
\begin{equation}
  S(\theta) - \eta \theta = \frac{1}{2} A_{ab} \theta_a \theta_b - \eta_a \theta_a
\end{equation}
and so completing the square we find
\begin{equation}
  S(\theta) - \eta \theta = \frac{1}{2} ((\theta_a - \eta_c (A^{-1})_{ca}) A_{ab}
  (\theta_b - \eta_d (A^{-1})_{db})) + \frac{1}{2} \eta_a (A^{-1})_{ab} \eta_b
\end{equation}
and so we use translational invariance to get
\begin{equation}
  \mathcal{Z}_0(\eta) = e^{-\frac{1}{2\hbar} \eta^T A^{-1} \eta}
  \mathcal{Z}_0(0)
\end{equation}
and propagator
\begin{equation}
  \langle \theta_a \theta_b \rangle = \frac{\hbar^2}{\mathcal{Z}_0} \partial_{\eta_a}
  \partial_{\eta_b} \mathcal{Z}_0(\eta) |_{\eta = 0} = \hbar (A^{-1})_{ab}
\end{equation}

\section{LSZ Reduction Formula}

(LSZ means Lehmann-Symanzik-Zimmerman) The LSZ reduction formula is a useful
tool, but for us it relates the concepts of scattering amplitudes, correlation
functions, and vacuum expectation values (VEVs - really the same as correlation
functions). It also introduces the normalisation of fields. Here, the lecturer
tries to keep the same conventions as David Tong notes, though he warns he may
not do so perfectly. In particular, that means that for $\hbar = 1$ we have
\begin{equation}
  \phi(x) = \int \frac{d^3k}{(2\pi)^3 2 E_k} \left( a(k) e^{-ik \cdot x} +
    a^\dagger(k) e^{ik \cdot x} \right)
\end{equation}
and use the mostly minus sign convention in Minkowski spacetime ($+---$) so
\begin{equation}
  k \cdot x = E_k t - k \cdot x
\end{equation}
and the relativistic normalisation for $a(k)$ mentioned below. We want to
express $a$ in terms of $\phi$ and as such we note that
\begin{align}
  \int d^3x e^{ik \cdot x} \phi(x) = \frac{1}{2E} a(k) +
  \frac{1}{2E} e^{2iEt} a^\dagger(-k) \\
  \int d^3x e^{ik \cdot x} \partial_a \phi(x) = \frac{-i}{2} a(k) +
  \frac{i}{2} e^{2iEt} a^\dagger(-k)
\end{align}

\subsubsection{Initial States}

To describe initial states then, for example in free theory with 1 particle, we
take
\begin{equation}
  \ket{k} = a^\dagger(k) \ket{\Omega}
\end{equation}
for the full vacuum state $\ket{\Omega}$ (Tong includes a discussion of full
vacuum states and another type) such that $a(k) \ket{\Omega} = 0 \forall k$ and
$\bra{\Omega} \ket{\Omega} = 1$) and we use the relativistic normalisation
\begin{equation}
  \bra{k} \ket{k'} = (2\pi)^3 (2E) \delta(k - k'), E = \sqrt{k^2 + m^2}
\end{equation}
[End of lecture 7]

\end{document}