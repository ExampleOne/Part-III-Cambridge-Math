\documentclass{article}
\usepackage[utf8]{inputenc}
\usepackage{mathtools}
\usepackage{amssymb}
\usepackage{graphicx}
\usepackage{listings}
\usepackage{float}
\usepackage{gensymb}
\usepackage{amsthm}
\usepackage{longtable}
\usepackage{adjustbox}
\usepackage{physics}
\usepackage{dsfont}
\usepackage{cancel}

\theoremstyle{definition}
\newtheorem{definition}{Definition}
\newtheorem{example}{Example}
\newtheorem{theorem}{Theorem}
\newtheorem{claim}{Claim}

\title{Quantum Information Theory}
\author{quinten tupker}
\date{January 22 2021 - \today}

\begin{document}

\maketitle

\section*{Introduction}

These notes are based on the course lectured by Professor Matthew Wingate in
Lent 2020. 
This was lectured online due to measures taken to counter the spread of Covid-19
in the UK. These are not necessarily an accurate representation of what was
lectures, and represent solely my personal notes on the content of the course,
combinged with probably, very very many personal notes and digressions... Of
course, any corrections/comments would be appreciated.

[the lecturer outlines the course] This course is an extension of the Michaelmas
Quantum Field Theory course that introduces renormalisation and the path
integral formulation of quantum field theory.

\section*{The Path Integral in Quantum Mechanics}

We start by reformulating the Schr\"{o}dinger equation as an integral equation,
which turns out to be a path integral. Anyways, starting with Schr\"{o}dinger's
equation for a Hamiltonian $H(x, p), [x, p] = i\hbar$ with

\begin{equation}
  H = \frac{p^2}{2m} + V(x)
\end{equation}

we have

\begin{equation}
  i\hbar \partial_t \ket{\psi(t)} = H \ket{\psi(t)} \implies \ket{\psi(t)} = e^{-i H t / \hbar}
  \ket{\psi(0)}
\end{equation}

where in the Schr\"{o}dinger picture the states evolve, but the operators remain
constant, and the wavefunction $\Psi(x, t) = \bra{x} \ket{\psi(t)}$. As such we
can rewrite our equation as

\begin{equation}
  \bra{x} H \ket{\psi(x)} = \left( \frac{-\hbar^2}{2m} \partial_x^2 + V(x) \right) \bra{x} \ket{\psi(t)}
\end{equation}

so we can write

\begin{align*}
  \Psi(x, t) &= \bra{x} \ket{\psi(t)} \\
             &= \bra{x} e^{-i H t / \hbar} \ket{\psi(0)} \\
             &= \int_{-\infty}^\infty dx_0 \bra{x} e^{-i H t / \hbar} \ket{x_0} \bra{x_0} \ket{\psi(0)} \\
             &= \int_{-\infty}^\infty dx_0 K(x, x_0, t) \Psi(x_0, 0)
\end{align*}

for \textbf{kernel} $K(x, x_0, t) = \bra{x} e^{-i H t / \hbar} \ket{x_0}$. Now,
if it is hard to calculate $K$ for large $t$, it can be beneficial to split this
into many intervals for many values of $t$, such as $0 = t_0 < t_1 < \dots < t_n
< t_{n + 1} = T$ leaving

\begin{equation}
  K(x, x_0, T) = \int_{-\infty}^\infty \prod_{r = 1}^n dx_r \bra{x_{r + 1}} e^{- iH(t_{r + 1} - t_r) / \hbar} \ket{x_r} \bra{x_1} e^{-iH (t_1 - t_0) / \hbar} \ket{0}
\end{equation}

which is in a sense an integral over all possible sequences of values of $x$. 

In free field theory ($V = 0$) this can be explicitly evaluated using a Gaussian
integral by rewriting things in the momentum basis as (use $\bra{x} \ket{p} =
e^{i px / \hbar}$)

\begin{align*}
  K_0(x, x', t) &= \bra{x} e^{\frac{-i p^2 t}{2m \hbar}} \int \frac{dp}{2\pi \hbar} \ket{p} \bra{p} \ket{x'} \\
                &= \int_{-\infty}^\infty \frac{dp}{2 \pi \hbar} e^{\frac{-ip^2 t}{2m \hbar}} e^{ip (x - x') / \hbar}\\
                &= e^{\frac{ip(x - x')^2}{2\hbar t}} \sqrt{\frac{m}{2\pi i \hbar t}}
\end{align*}

where we note that the limit as $t \to 0$ is $\delta(x - x')$ which indeed
matches $\bra{x} \ket{x'} = \delta(x - x')$ as expected.

Now in an interacting theory, we struggle with the Baker-Campbell-Hausdorff fact
that $e^A e^B \neq e^{A + B}$ so using Suzuki-Trotter we separate into steps
size $t_{r + 1} - t_r = \delta t << T$ meaning that

\begin{equation}
  e^{-iH \delta t / \hbar} \approx e^{\frac{-ip^2 \delta t}{2m \hbar}} e^{\frac{-i V(x) \delta}{\hbar}} (1 + O(\delta t^2))
\end{equation}

so for $T = n \delta t$ we find that

\begin{equation}
  K(x, x_0, T) = \int \prod_{r = 1}^n dx_r \left( \frac{m}{2\pi i \hbar \delta t} \right)^{\frac{n + 1}{2}}
  e^{i \sum_{r = 0}^n \left( \frac{m}{2 \hbar} \left( \frac{x_{r + 1} - x_r}{\delta t} \right)^2
      - V(x_r) / \hbar \right) \delta t}
\end{equation}

which in the limit $n \to \infty, \delta t \to 0$ while keeping $T$ constant
leaves

\begin{equation}
  \frac{1}{\hbar} \int_0^T dt \left( \frac{1}{2}m \dot{x}^2 - V(x) \right) = \int_0^T dt L(x, \dot{x}) = S
\end{equation}

for classical Lagrangian $L$ and action $S$. This is what we refer to as a path
integral or function integral:

\begin{equation}
  K(x, x_0, t) = \int \mathcal{D}x e^{i S / \hbar}
\end{equation}

where $\mathcal{D} x$ is the limit describd above. Of course, many questions
about the existence and uniqueness, etc. of such limits exists, and in fact
often this limit does not exist, but in the cases we are interested it, it works
well enough...

\end{document}