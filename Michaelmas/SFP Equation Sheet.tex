\documentclass{article}
\usepackage[utf8]{inputenc}
\usepackage{mathtools}
\usepackage{amssymb}
\usepackage{graphicx}
\usepackage{listings}
\usepackage{float}
\usepackage{gensymb}
\usepackage{amsthm}
\usepackage{longtable}
\usepackage{adjustbox}
\usepackage{physics}

\theoremstyle{definition}
\newtheorem{definition}{Definition}
\newtheorem{example}{Example}
\newtheorem{theorem}{Theorem}

\title{Symmetries Fields and Particles Equation Sheet}
\author{quinten tupker}
\date{November 10 2020 - \today}

\begin{document}

\maketitle

Here are some useful equations

\begin{table}[H]
  \caption{Equation Sheet}
  \begin{adjustbox}{center}
    \begin{tabular}{|p{5cm}|p{10cm}|p{5cm}|}
      \hline
      \label{equations_1}
      Name/Description & Equation & Remarks \\ \hline
      Pauli Matrices & $\sigma_1 = \begin{pmatrix} 0 & 1 \\ 1 & 0 \end{pmatrix}, \sigma_2 = \begin{pmatrix} 0 & -i \\ i & 0 \end{pmatrix}, \sigma_3 = \begin{pmatrix} 1 & 0 \\ 0 & -1 \end{pmatrix}$ & Together with the identity these form a basis for the space of traceless 2 by 2 matrices as a real vector space. Also, these are orthogonal under the trace matrix inner product $\tr(AB)$. \\ \hline
      Pauli Matrix product & $\sigma^i \sigma^j = \delta^{ij} I + i \epsilon^{ijk} \sigma^k$ & \\ \hline
    \end{tabular}
  \end{adjustbox}
\end{table}

\end{document}