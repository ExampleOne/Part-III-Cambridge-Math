\documentclass{article}
\usepackage[utf8]{inputenc}
\usepackage{mathtools}
\usepackage{amssymb}
\usepackage{graphicx}
\usepackage{listings}
\usepackage{float}
\usepackage{gensymb}
\usepackage{amsthm}
\usepackage{longtable}
\usepackage{adjustbox}
\usepackage{physics}

\theoremstyle{definition}
\newtheorem{definition}{Definition}
\newtheorem{example}{Example}
\newtheorem{theorem}{Theorem}

\title{Quantum Field Theory Equation Sheet}
\author{quinten tupker}
\date{October 7 2020 - \today}

\begin{document}

\maketitle

Here are some useful equations

\begin{table}[H]
  \caption{Equation Sheet}
  \begin{adjustbox}{center}
    \begin{tabular}{|p{5cm}|p{10cm}|p{5cm}|}
      \hline
      \label{equations_1}
      Name/Description & Equation & Remarks \\
      Noether conserved current & $j^\mu = \partial_{\partial_\mu \phi} \mathcal{L} \delta \phi - F^\mu$ & here $\mathcal{L}(x + \delta x) = \mathcal{L} + \delta x \partial_\mu F^\mu$, $\partial_\mu j^\mu = 0$ \\ \hline
      The conserved charge arising from a conserved current & $Q = \int d^3x j^0$ & \\ \hline
      The Energy-Momentum Tensor & $T^\mu_\nu = \partial_{\partial_\mu \phi} \mathcal{L} \partial_\nu \phi - \delta^\mu_\nu \mathcal{L}$ & This is the Noether current under translation. This tensor can always be chosen to be symmetric. It is a Noether current, so conserved as $\partial_\mu T^{\mu \nu} = 0$ \\ \hline
    \end{tabular}
  \end{adjustbox}
\end{table}

\end{document}