\documentclass{article}
\usepackage[utf8]{inputenc}
\usepackage{mathtools}
\usepackage{amssymb}
\usepackage{graphicx}
\usepackage{listings}
\usepackage{float}
\usepackage{gensymb}
\usepackage{amsthm}
\usepackage{longtable}
\usepackage{adjustbox}
\usepackage{physics}
\usepackage{cancel}

\theoremstyle{definition}
\newtheorem{definition}{Definition}
\newtheorem{example}{Example}
\newtheorem{theorem}{Theorem}

\title{Quantum Field Theory Equation Sheet}
\author{quinten tupker}
\date{October 7 2020 - \today}

\begin{document}

\maketitle

Here are some useful equations

\begin{table}[H]
  \caption{Equation Sheet}
  \begin{adjustbox}{center}
    \begin{tabular}{|p{5cm}|p{10cm}|p{5cm}|}
      \hline
      \label{equations_1}
      Name/Description & Equation & Remarks \\ \hline
      Noether conserved current & $j^\mu = \partial_{\partial_\mu \phi} \mathcal{L} \delta \phi - F^\mu$ & here $\mathcal{L}(x + \delta x) = \mathcal{L} + \delta x \partial_\mu F^\mu$, $\partial_\mu j^\mu = 0$ \\ \hline
      The conserved charge arising from a conserved current & $Q = \int d^3x j^0$ & \\ \hline
      The Energy-Momentum Tensor & $T^\mu_\nu = \partial_{\partial_\mu \phi} \mathcal{L} \partial_\nu \phi - \delta^\mu_\nu \mathcal{L}$ & This is the Noether current under translation. This tensor can always be chosen to be symmetric. It is a Noether current, so conserved as $\partial_\mu T^{\mu \nu} = 0$ \\ \hline
      Ladder Operators & $[a_p, a_q^\dagger] = (2\pi)^3 \delta(p - q)$ & \\ \hline
      Field Operator & $\phi = \int \frac{d^3p}{(2\pi)^3} \frac{1}{\sqrt{2\omega_p}} (a_p e^{i p \cdot x} + a_p^\dagger e^{-i p \cdot x})$ & \\ \hline
      Momentum Operator & $\pi = \int \frac{d^3p}{(2\pi)^3} (-i) \sqrt{\frac{\omega_p}{2}} (a_p e^{i p \cdot x} - a_p^\dagger e^{-ip \cdot x})$ & \\ \hline
      Dirac Equation & $(i \gamma^\mu \partial_\mu - m) \cdot \psi = (i \cancel{\partial} - m) \cdot \psi = 0$ & \\ \hline
      Chiral represntation and Clifford Algebra & $\{\gamma^\mu, \gamma^\nu \} = 2\eta^{\mu \nu}I, \{\gamma^5, \gamma^\mu\} = 0, (\gamma)^2 = 0, \gamma^5 = i\gamma^0 \gamma^1 \gamma^2 \gamma^3$, $\gamma^0 = \begin{pmatrix} & I \\ I & \end{pmatrix}$, $\gamma^i = \begin{pmatrix} & \sigma^i \\ \sigma^i & \end{pmatrix}$, $\gamma^5 = \begin{pmatrix} I & \\ & -I \end{pmatrix}$ & \\ \hline
      General Solution to Dirac Equation &
      positive frequencies: $\psi(x) = u(p) e^{-i p \cdot x}, u^s(p) = \begin{pmatrix} \sqrt{p \cdot \sigma} \xi^s \\ \sqrt{p \cdot \bar{\sigma}} \xi^s \end{pmatrix}$,
      for negative frequencies $\psi(x) = v(p) e^{ip \cdot x}, v^s(p) = \begin{pmatrix} \sqrt{p \cdot \sigma} \eta^s \\ -\sqrt{p \cdot \bar{\sigma}} \eta^s \end{pmatrix}$ &
      Here $\xi^r, \eta^s$ form orthonormal bases for $\mathbb{C}^2$ and
      $\sqrt{p \cdot \sigma} = \sqrt{m} e^{\chi \cdot \sigma / 2}, \sqrt{p \cdot \bar{\sigma}} = \sqrt{m} e^{-\chi \cdot \sigma}$ \\ \hline
    \end{tabular}
  \end{adjustbox}
\end{table}

\end{document}