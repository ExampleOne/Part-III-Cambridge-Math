\documentclass{article}
\usepackage[utf8]{inputenc}
\usepackage{mathtools}
\usepackage{amssymb}
\usepackage{graphicx}
\usepackage{listings}
\usepackage{float}
\usepackage{gensymb}
\usepackage{amsthm}
\usepackage{longtable}
\usepackage{adjustbox}
\usepackage{physics}

\theoremstyle{definition}
\newtheorem{definition}{Definition}
\newtheorem{proposition}{Proposition}

\title{Statistical Field Theory}
\author{quinten tupker}
\date{October 9 2020 - \today}

\begin{document}

\maketitle

\section*{Introduction}

These notes are based on the course lectured by Dr Christopher E Thomas in
Michaelmas 2020. Due to the measures taken in the UK to limit the spread of
Covid-19, these lectures were delivered online. These are not meant to be an
accurate representation of what was lectures. They solely represent a mix of
what I thought was the most important part of the course, mixed in with many
(many) personal remarks, comments and digressions... Of course, any
corrections/comments are appreciated.

Statistical Field Theory is an extension of statistical physics. It assumes one
is familiar with statistical physics, and in particular, focuses on the study on
phase transitions. This course in particular follows David Tong's notes quite
closely.

\section{From Spin to Fields}

So why do we consider fields? Here we look at a model that shows the origin of
this connection.

\subsection{The Ising model}

The Ising model studies a lattice where each point on the lattice is assigned a
spin $S_i = \pm$. Consequently, the energy of the lattice is

$$ E = -B \sum_i S_i - J \sum_{<i, j>} S_i S_j $$

where $B$ is the external magnetic field strength, $J$ is the strength of
neighbour-neighbour interactions and $<i, j>$ is any nearest neighbour pair.
When $J > 0$, states tend to align (ferromagnetic behaviour) whereas if $J < 0$
states will prefer not to align (anti-ferromagnetic behaviour). We will focus on
the $J > 0$ case. But of course, due to heat there some statistical randomness,
and we want to include that. As such, we consider the canonical ensemble with
$\mathbb{P} (S_i) = e^{-\beta E(S_i)} / Z$ where $Z$ is the partition function,
$\beta = 1 / T$, and the Boltzmann constant $k_B = 1$. As ever in statistical
physics, we can derive everything from the partition function. Particularly
important in our case is the Free Energy, $F = \langle E \rangle - TS = -T \ln
(Z)$, and $dF = -S dT - p dV - M dB$ (so we can find S, p and M from F as well).

In statistical physics, we are particularly interested in the equilibrium state
(which occurs at the minimum of the free energy when temperature is constant),
and since we are looking at a magnetic system, we are particularly interested in
the equilibrium magnetisation. This can be calculated as

$$ m = \frac{1}{N} \sum_i \langle S_i \rangle = \frac{1}{N \beta} \partial_B
\ln(Z) $$

Now all that remains is to calculate $Z$ to find $m$. Unfortunately, in
dimension 3 or greater this is impossible (how impossible?), and it is still
hard in lower dimension. Consequently we take a different approach using the
so-called ``effective free energy'', which is defined such that

$$ \sum_m \sum_{\{S_i\} | m} e^{-\beta E[S_i]} = \sum_m e^{-\beta F(m)} $$

where $F(m)$ is the effective free energy. Since we can assume $N$ is large
(around $10^{23}$), we can then write18

$$ Z = N / 2 \int_{-1}^1 dm \, e^{-\beta F(m)}  = N / 2 \int_{-1}^1 dm \,
e^{-\beta N f(m)}$$

where $f(m) = F(m) / N$. From here, we can calculate the equilibrium field by
considering that since $N$ is large, the value contributing the most to the
integral is where $\partial_m f = 0$, and this is how the equilibrium $m$ is
calculated. This approach is called the \textbf{steepest descent approximation},
and we find here that $F_{\text{thermodynamic}} \approx F(m_{\text{min}})$.

This is all very well and nice, but the issue is that we still don't know how to
calculate $F(m)$, and it turns out that this is about as hard as calculating
$Z$. As such, we use the \textbf{mean field approximation}

$$ E \approx -B \sum_i m - J \sum_{<i, j>} m^2 = -BNm - \frac{1}{2} N J q m^2 $$

where $q = 2 \text{dim(space)}$ for a cubic latice in a space dimension dim.
Now,

$$ e^{-\beta N f(m)} = \sum e^{-\beta E(S_i) \approx \Omega(m) e^{-\beta
    E(m)}} $$

so in order to find $f(m)$ we need to know $\Omega(m)$ which is the number of
ways for a given energy state to occur, but since $m$ depends only on the number
of positive and negative spins states, $N_\uparrow, N_\downarrow$, and in
particular $N = N_\uparrow + N_\downarrow$ we see the number of total states is

$$ \ln(\Omega) = \ln \binom{N}{N_\uparrow} \approx N (\ln(2) - \frac{1}{2}(1 +
m) \ln(1 + m) - \frac{1}{2} (1 - m) \ln(1 - m)) $$

by Stirling's approximation. Consequently we find that

$$ f(m) \approx -Bm - \frac{1}{2} J q m^2 - \frac{1}{\beta} (\ln(2) -
\frac{1}{2} (1 + m) \ln(1 + m) - \frac{1}{2} (1 - m) \ln(1 - m)). $$

From here  by taking $\partial_m f = 0$, we find

$$ \beta(B + Jqm) = \frac{1}{2} \ln \left( \frac{1 + m}{1 - m} \right) $$

or equivalently

$$ m = \tanh(\beta (B + Jqm)). $$

From here, we can solve implicitly, but there is another approach we can take,
which is the way we will go about this. Just as a side note, we can see the
reason this is called the mean field approximation here as well: it is as if $m$
just shifts to external field to $B_{\text{eff}} = B + Jqm$. [End of lecture 1.]

\end{document}