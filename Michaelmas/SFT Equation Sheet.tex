\documentclass{article}
\usepackage[utf8]{inputenc}
\usepackage{mathtools}
\usepackage{amssymb}
\usepackage{graphicx}
\usepackage{listings}
\usepackage{float}
\usepackage{gensymb}
\usepackage{amsthm}
\usepackage{longtable}
\usepackage{adjustbox}
\usepackage{physics}

\theoremstyle{definition}
\newtheorem{definition}{Definition}
\newtheorem{example}{Example}
\newtheorem{theorem}{Theorem}

\title{Statistical Field Theory Equation Sheet}
\author{quinten tupker}
\date{October 7 2020 - \today}

\begin{document}

\maketitle

Here are some useful equations

\begin{table}[H]
  \caption{Equation Sheet}
  \begin{adjustbox}{center}
    \begin{tabular}{|p{5cm}|p{10cm}|p{5cm}|}
      \hline
      \label{equations_1}
      Name/Description & Equation & Remarks \\ \hline
      Equilibrium Magnetisation in the Ising Model & $m = \frac{1}{N \beta} \partial_B \ln(Z)$ & \\ \hline
      Heat Capacity $C$ & $C = \partial_T \langle E \rangle = \beta^2 \partial_\beta^2 \ln(Z)$ & For critical exponents one can use that $c \sim \partial_T^2 f$ \\ \hline
      Magnetic susceptibility & $\chi = \partial_B m |_{B = 0}$ & \\ \hline
      Calculating Critical Exponents & $m \sim (Tc - T)^\beta$ for $T < T_c$ when $B = 0$, $c sim c_\pm |T - T_c|^{-\alpha}$ when $B=0$ again, $\chi \sim |T - T|^{-\gamma}$ when $m$ is small so ignore higher order terms, $m \sim B^{1 / \delta}$ when $T \approx T_c$ & \\ \hline
    \end{tabular}
  \end{adjustbox}
\end{table}

\end{document}