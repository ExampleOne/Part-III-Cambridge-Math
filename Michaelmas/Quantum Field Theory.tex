\documentclass{article}
\usepackage[utf8]{inputenc}
\usepackage{mathtools}
\usepackage{amssymb}
\usepackage{graphicx}
\usepackage{listings}
\usepackage{float}
\usepackage{gensymb}
\usepackage{amsthm}
\usepackage{longtable}
\usepackage{adjustbox}
\usepackage{physics}

\theoremstyle{definition}
\newtheorem{definition}{Definition}
\newtheorem{example}{Example}
\newtheorem{theorem}{Theorem}

\title{Quantum Field Theory}
\author{quinten tupker}
\date{October 7 2020 - \today}

\begin{document}

\maketitle

\section*{Introduction}

These notes are based on the course lectured by Professor Nicholas Dorey in Michaelmas 2020.
This was lectured online due to measures taken to counter the spread of Covid-19
in the UK. These are not necessarily an accurate representation of what was
lectures, and represent solely my personal notes on the content of the course,
combinged with probably, very very many personal notes and digressions... Of
course, any corrections/comments would be appreciated.

But, let's actually introduce the content of this course. What is quantum field
theory? Quantum field theory (QFT) essentially succeeds in merging special
relativity
and quantum mechanics. Why is this so difficult? The first relativistic theory
was electromagnetism, and the biggest idea that was introduced, and what truly
set it apart from older theories was the idea of fields. In Newtonian theory, and much of what
followed, the protagonist of the theory was a particle, or if not a particle, at
least a body of some kind. Field theory changed. Now, there was a new
protoganist: the field. What difference does that make, architectually? Firstly,
the field theory is often simpler. But more importantly, the biggest structural
difference is that field theory excels in describing ``delayed interactions.''
When the particle is the progagonist, it is very difficult describe theories
where forces are not instantaneous. Field theory avoids this. All interactions
are made through the field, through which they propogate through space. Now,
delayed responses become natural. In electromagnetism, the simplest expression
thereof is electromagnetic waves: light. 

What does this have to do with relativity? Well, as soon as high speeds become
relevant, forces can no longer be considered instantaneous. As such, it is
difficult to keep using particles as the protoganist of these theories.
Consequently, the natural step is to make, instead of particles, fields the
protoganist of this new quantum theory we are developing.

That is the goal of QFT. There is one important consequence though, once
particles are no longer the protagonists of the theory. That is that particle
number no longer has to be conserved. In the most elegant fashion, by removing
the supremacy of the ``particle'' in our theory, and replacing it with the more
powerful notion of the field, particles merely become phenomenon to be observed,
and tools of analysis. In this context, it is only natural that particle number
is no longer conserved. Whereas before, the wavefunction was often associated
with a wave-particle like object, now the wavefunction (which is a field)
describes a multiparticle state. Well, really it describes the field, and the
multiparticle state is something that can be deduced from it. Somehow, although
this is just the beginning of the course, I feel that that's not that important
anymore. It is deeply intriguing though, how the imposition of boundary
conditions somehow forces a degree of discreteness onto this theory...

Well then, the overall architecture is more or less the same as standard quantum
theory. It is probabilistic, and we assume a degree of symmetry under
boosts, and rotations (isotropy and translation invariance). The fundamental
approach to making predictions still boils down to the same calculation:
evaluating

$$ A_{i \to f} = \bra{f} e^{i H T} \ket{i} $$

for probability amplitude $A$, initial state $i$, final state $f$, Hamiltonian
(time translation generator) $H$, and time interval $T$.

There are two caveates with most of these field theories, though. Firstly, they
have not been mathematically formalised, so often there are areas that are
somewhat ambiguous. Secondly, contributing to this ambiguity, many of the sums
are divergent, so the meaning of some calculations can really be somewhat
ambiguous... How curious! I'd like to think about this a bit more...

\section{Preliminaries}

Anyways, getting down to business. We'll be mostly using natural units during
this course. That means that $c = \hbar = 1$, and these can be added back into
the calculation using dimensional analysis. The effect of this, is that the only
unit used throughout all calculations is really a unit of mass-energy. As such,
all quantities scale by a power of the unit of energy. 

\begin{definition}[dimension of $X$]
  Denoted $[X]$, this is $\delta$, such that for unit of mass-energy $M$, $X$
  scales as $M^\delta$. $\delta$ may also be called the scaling or the
  engineering dimension of $X$.
\end{definition}

Also, for special relativity we use the convention that we are working on
Minkowsky space-time $\mathbb{R}^{3, 1}$, with metric tensor

$$ \eta_{\mu \nu} = \text{diag}(1, -1, -1, -1). $$

\section{Classical Field Theory}

Fortunately for me, we are starting with a description of classical field
theory, and we are starting, very simply, with scalar fields.

\begin{definition}[Scalar Field]
  $\phi(t, \underline{x}) = \phi(x) : \mathbb{R}^{3, 1} \to \mathbf{R}$ is a sclar field
  if it is Lorentz invariant, meaning that it follows the transformation rule
  $$ \phi(x) \to \phi(\Lambda^{-1}(x)) $$
  Here the domain is called the spacetime, and the codomain is called the
  \textbf{field space}. These may, and will be replaced with other spaces.
\end{definition}

Changing the spacetime here corresponds to implementing gravity in some way or
another, by changing the manifold we are working on. The field space corresponds
to the complexity of what is being descibed. Since we will be describing
multi-particle states with our wavefunction, this will become significantly more
complex. And, I intentionally left the description of what it means to be
Lorentz invariant a bit vague, since, well, in our case, it just means being
invariant under the Lorentz transformations, which is the group of
linear transformations that preserves the Minkowsky metric (ie. the set of
matrices such that $\Lambda \eta \Lambda^T = \eta$). But really, while linearity
makes a lot of sense in the context of linear Minkowsky space, I doubt (though I
have no familiarity with this area) this remains the case when we are on an
arbitrary manifold, which happens when we consider general relativity. As such,
I prefer to think of $\Lambda$ as any arbitrary invertible map on the manifold,
corresponding to the symmetries we impose. 

The difficulties that arise in comibing quantum theory with general relativity
are also quite clear. It does seem tremendously difficult. If we do not simply
assuming that general relativity simply bends space, which I will assume is not
entirely the case, or else I feel a theory reconciling the two would have
already been developed long ago, in spite of the tremendous difficulty of the
calculations involved, and it also does not seem to make much sense of Hawking
radiation, since in general relativity, the space at the centre of a black hole
truly is cut off... Nevertheless, from what I've heard, if you want to turn
gravity into a quantised force with mediator particle, then somehow the
protagonist of that theory would be not only be a field, but somehow span the
space of possible manifolds as well. Purely intuitively, I would imagine that we
would be getting fields of the form $\phi : \mathcal{D} \to V$ where
$\mathcal{D}$ is an object that stitches many manifolds together. Brrr... I have
not thought too deeply about this, but that does seem like a truly terrifying
object indeed! Or perhaps not quite. Hm, it might be worth thinking a bit more
about this...

Anyways, I also wanted to remark that having fields transform as
$\phi(\Lambda^{-1}x)$ is more or less an arbitrary definition that is called the
``active'' definition of field transformations. Oh well, you can assign some
intuition to it, but it is more or less convention to use the inverse of the
matrix instead of the matrix itself.

Extending our notion of fields to vector fields, first note that notation wise,
we use $\partial_\mu \phi = \frac{\partial \phi}{\partial x^\mu}$, and we define

\begin{definition}[$\phi^\mu$ transforms as a vector field]
  if
  $$ \phi \to \Lambda^\mu_\nu \phi^\nu (\lambda^{-1} x) $$
\end{definition}

This is just the transformation rule for rank 1 tensors, so is nothing
particularly remarkable. The only remarkable part is that, as a result
$\partial^\mu \phi$ transformas as vector, and so the following becomes a rank 0
tensor (ie, a scalar field) $\partial^\mu \phi \partial_\mu \phi$.

Well, that ends [lecture 1], for those of you interested. 

\subsection{The Lagrangian}

We will review some of the tools from classical mechanics that we are using. The
first is the Lagrangian. There are three advantages to using the lagrangian
here:

\begin{itemize}
\item they are independent of coordinates
\item the symmetries of the system can be easily expressed
\item the path integral formulation follows immediately
\end{itemize}

although the Lagrangian is used much more heavily in the Advanced Quantum Field
Theory course than the current course. Anyways, we review the Lagrangian, except
that now we implement it on fields, so $L = L(\phi, \partial_\mu \phi)$ for
(scalar) fields $\phi$, and we make three requirements of our action/Lagrangian

\begin{itemize}
\item The action $S$ is Lorentz invariant
\item we require locality (see below)
\item we have at most a second order time derivative
\end{itemize}

Locality essentially means that $L = \int dx^3 \, \mathcal{L}(\phi(x),
\partial_\mu \phi(x))$, so the Lagrangian has an associated local Lagrangian
density $\mathcal{L}$. From hereon, this $\mathcal{L}$ will usually be referred
to as the Lagrangian instead. We note that the action (over an infinite time
scale) may now be expressed as

$$ S = \int dx^4 \mathcal{L}(\phi, \partial_\mu \phi) $$

Now, Lorentz invariance essentially means that the Lagrangian density is a
scalar field (a tensor really), meaning that it transforms as $\mathcal{L}(x)
\mapsto \mathcal{L}(\Lambda^{-1} x)$. This corresponds to the action being
Lorentz invariant since the Jacobian of a Lorentz transformation (the
determinant) is 1, so the relevant change of variables, the action does not
change.

Finally, using no more than a second order time derivative in the context of
relativity means using no more than a second order derivative of any kind, and
since we also are rank-0 tensors, we in fact, cannot have first order
derivatives either. Consequently, we can write the general form of the
Lagrangian as 

$$ \mathcal{L} = \frac{1}{2} F(\phi) \partial_\mu \phi \partial^\mu \phi - V(\phi) $$

(note that $\phi \partial_\mu \partial^\mu \phi$ is related to the above by
integration by parts so can be safely ignored). Also, in practice in quantum
theory we may neglect $F(\phi)$ leaving us with quite a simple general form.

To make all this work, we apply the principle of least action, which gives us
the Euler Lagrange equation

$$ \partial_\mu \partial_{\partial_mu \phi} \mathcal{L} = \partial_\phi
\mathcal{L} $$

An important special case to consider is the case when the potential is
quadratic, so $V(\phi) = \frac{1}{2} M^2 \phi^2$, which means the Euler-Lagrange
case is linear, leaving us with the so-called \textbf{Klein-Gordon Equation}

$$ \partial_\mu \partial^\mu \phi + M^2 \phi = 0 $$

As one might expect, since in Minkowski spacetime $\partial_\mu \partial^\mu =
\partial_t^2 - \nabla^2$ we get wavelike solutions

$$ \phi \sim e^{ix \cdot p} $$

where $x \cdot p = \omega t - \vec{k} \cdot \vec{x}$, and the dispersion
relation requires $\omega_k = \sqrt{k^2 + M^2}$.

On a philosophical note, I was wondering what difference using the principle of
least action makes compared to just Newton's equations (on a philosophical level
- it is obviously more practical), and as such I was wondering to what extent
the Lagrangian formulation in a sense is just a tensor formulation of Newton's
equations? 

I also was wondering what locality means. When discussing life, I have remarked
that really as long as one is not starving, etc., reality is little more than a
medium for communication between people. This is certainly the case for
particles, and objects, which do not have to worry about starvation, etc. But
then I wonder, do particles really not need to worry about starvation? Many
particles decay after all, although I doubt that has to do with a kind of
starvation of any kind... [End of lecture 2]

\subsection{Maxwell's Theory}

Maxwell's theory uses a 4-vector potential $A^\mu = (\phi, \vec{A})$. This being
a rank 1 tensor means it transforms as

$$ A^\mu(x) \mapsto \Lambda^\mu_\nu A^\nu(\Lambda^{-1} x) $$

In electromagnetism, the \textbf{field strenght tensor} is given by

$$ F^{\mu \nu} = \partial^\mu A^\nu - \partial^\nu A^\mu $$

and as a tensor, it transforms as appropriate. But it satisfies a further
condition that it is invariant under Gauge transformations $A^\mu \mapsto A^\mu
+ \partial^\mu \lambda$. This leads to the \textbf{Bianchi identity}

$$ \partial_\lambda F_{\mu \nu} + \partial_\mu F_{\nu \lambda} + \partial_\nu
F_{\lambda \mu} = 0 $$

which captures 2 out of Maxwell's 4 equations. The other two arise from the
principle of least action applied to the \textbf{Maxwell Lagrangian}

$$ \mathcal{L} = -\frac{1}{4} F_{\mu \nu} F^{\mu \nu} $$

We should note that there really are not that many options of Lagrangians here
once one introduces the standard assumptions in addition to assuming it must be
expressed in terms of physical quantities (the only candidate is the field
strength vector used above)... Writing

$$ \mathcal{L} = -\frac{1}{2} \partial_\mu A_\nu \partial^\mu A^\nu +
\frac{1}{2} (\partial_\mu A^\mu)^2 \eta^{\mu \nu} $$

gives us (after Euler-Lagrange)

$$ \partial_\mu F^{\mu \nu} = 0 $$

which provides us with the rest of the Maxwell equations.

\subsection{Symmetries in QFT}

Symmetries (see the Symmetries, Fields, and Particles (SFP) course) are variations of
fields that leave the action invariant. Symmetries have several effects in
physics. Firstly, by Noether's theorem, they lead to conservation laws.
Secondly, they restrict the form of the Lagrangian. In particular, Gauge
symmetry is very powerful due to strong restrictions it puts on the form of the
Lagrangian. 

Common symmetries include (See SFP) translation invariance and Lorentrz
transformation invariance. Another symmetry occurs when either $m = 0, V = 0$ or
the action is propoertional to the field. In this case we get an additional
``scale invariance'' where $x^\mu \mapsto \lambda x^\mu$ and $\phi \mapsto
\lambda^{-\Delta} \phi(\lambda^{-1}x)$ for engineering constant $\Delta$.

Internal symmetries like charge, flavour or colour conservation also occur,
further restricting the theory. These are not related to the space, and are
somehow better expressed by the quantity they conserve. Another small curious
symmetry is the following:

\begin{example}
For complex scalar fields with $\mathcal{L} = \partial_\mu \psi^* \partial^\mu
\psi - V(|\psi|^2)$ the following is also a symmetry: $\psi \mapsto e^{i \alpha}
\psi$ for real $\alpha$. The lecturer does not go into any more detail about
this.
\end{example}

Finally, we note that for continuous symmetries, which form Lie groups, we can
take the set of elements of the group ``near'' the identity, $g = e^{\alpha X}$
where $X$ sits in the Lie algebra of $G$. Working to 1st order we now may write

$$ g \cdot \psi \mapsto \psi + \delta \psi = \psi + \alpha X \psi $$

to first order. [End of lecture 3]

Just as a remark on terminology, a Lorentz transformation is a linear
transformation preserving the Minkowski metric. On the other hand, a
\textbf{proper Lorentz transformation} is a Lorentz transformation with
determinant 1 (which ensures causality). 
Our goal now is to work towards Noether's theorem. Noether's theorem states that \begin{theorem}[Noether's Theorem] For every continuous symmetry, there exists a conserved current, given by
  $$ j^\mu = \partial_{\partial_\mu \phi} \mathcal{L} \delta \phi -
  F^\mu(\phi(x)) $$
  where $\delta \phi$ is the variation in $\phi$ and $\delta \mathcal{\lambda} =
  F^\mu$.
  Conservation here means that $\partial_\mu j^\mu = 0$.
\end{theorem}

The conserved quantity associated with the conserved current then can be shown
to be $Q = \int dx^3 j^0$ since

\begin{align*}
  \frac{d}{dt} Q
  &= \int dx^3 \partial_t j^0 \\
  &= - \int dx^3 \nabla \cdot J \\
  &= \int_S dS \cdot J \\
  &= 0
\end{align*}

A good example is $j^\mu = (\rho, J)$ for charge density and current in
electromagnetism. Anyways, here is our ``proof'' of Noether's theorem:

\begin{proof}
  Using the Euler-Lagrange equation applied to the field $\phi$ we see that
  \begin{align*}
    \delta \mathcal{L}
    &= \partial_\phi \mathcal{L} \delta \phi + \partial_{\partial_\mu \phi} \mathcal{L} \delta \partial_\mu \phi \\
    &= (\partial_\phi \mathcal{L} - \partial_\mu (\partial_{\partial_\mu \phi} \mathcal{L})) \delta \phi + \partial_\mu (\partial_{\partial_\mu \phi} \mathcal{L} \delta \phi) \\
    &= \partial_\mu (\partial_{\partial_\mu \phi} \mathcal{L} \delta \phi)
  \end{align*}

  Then using our definition of $j^\mu$ as above we see that $\partial_\mu j^\mu
  = 0$.
\end{proof}

Here it is assumed that $\delta \mathcal{L} = \partial_\mu F^\mu$ which means we
can apply Stokes' theorem and integrate on the surface only. When we do so,
$F^\mu$ becomes that surface term we integrate with. But Euler-Lagrange also
means that our expression above, for the first term in the conserved current, is
also a surface term (often determined by the boundary conditions) for $\delta
\mathcal{L}$. So what exactly is the difference? Is it that the first term
depends on the variation whereas the second does not?

[End of lecture 4 - this bit needs rewriting]

[Start of Noether's theorem rewrite]

Alright, so now to the main topic of this lecture: Noether's theorem. Noether's
theorem states that for every continuous symmetry we get a conserved current
(and quantity), which she gives an equation for. Now how do we get that?

Firstly, we observe that for a Lorentz transformation

$$ \Lambda \approx I + s\Omega $$

where $\Omega$ is antisymmetric (to ensure we preserve the Minkowski metric). If
that the case, we see that if we tree $\mathcal{L}$ as a scalar field
$\mathcal{L}(x)$ instead of a function of the field, then

$$ \delta \mathcal{L} = - s\Omega^\mu_\nu x^\nu \partial_\mu(\mathcal{L}(x)) =
-s\Omega^\mu_\nu \partial_\mu (x^\mu \mathcal{L}(x)). $$

This last equality only holds because $\Omega$ is asymmetric (so its diagonal is
all zeros). But importantly, we see that $\delta \mathcal{L}$ can be written as
a total derivative when $x$ is varied.

Now, we vary the same $\mathcal{L}$ but instead of varying with respect to $x$,
we vary with respect to the field $\phi$. Doing so, and assuming Euler-Lagrange
we find that:

\begin{align*}
  \delta \mathcal{L}
  &= \partial_\phi \mathcal{L} \delta \phi + \partial_{\partial_\mu \phi} \mathcal{L} \delta \partial_\mu \phi \\
  &= (\partial_\phi \mathcal{L} - \partial_\mu (\partial_{\partial_\mu \phi} \mathcal{L})) \delta \phi + \partial_\mu (\partial_{\partial_\mu \phi} \mathcal{L} \delta \phi) \\
  &= \partial_\mu (\partial_{\partial_\mu \phi} \mathcal{L} \delta \phi)
\end{align*}

Remarkably, we find once again that we may write $\delta \mathcal{L}$ as a total
derivative. Let us now denote these variations as $\delta_x \mathcal{L}$ for the
first, and $\delta_\phi \mathcal{L}$ for the second. Now, since we have a
symmetry, we know that the action is unchanged when we apply a Lorentz
transformation. In particular, we know that the ``partial variation'' of
$\mathcal{L}$ with respect to $x$ alone, but ignoring $\phi$ should be
``constant.'' What do we mean by constant? The partial derivative with respect
to space is a 4-vector, so really we mean that the Minkowski metric is
unchanged. In other words, to first order the quantity

$$ v_M \cdot \partial_x \delta \mathcal{L} = 0 $$

where we use $\partial_x \delta$ to denote this ``pure'' partial derivative and

$$
v_M =
\begin{pmatrix}
  -1 \\
  1 \\
  1 \\
  1
\end{pmatrix}
$$

is the ``Minkowski vector'' (the first order derivative of the Minkowski metric
in a sense). This would mean that the Minkowski metric is unchanged under a pure
variation of spacetime. So how do we write this pure variation of space time?
We notice that $\delta \partial_x \mathcal{L} = \delta_x \mathcal{L} -
\delta_\phi \mathcal{L}$ since in a sense $\delta_x \mathcal{L}$ is the total
derivative of $\mathcal{L}$ with respect to $x$, and $\delta_\phi \mathcal{L}$
is the total derivative of $\mathcal{L}$ with respect to $\phi$, and so, by the
chain rule, what is left over is the partial variation with respect to $x$.

This is exactly what Noether's theorem refers to. Noether's theorem defines the
\textbf{conserved current} to be

$$ j^\mu = \partial_{\partial_\mu \phi} \mathcal{L} \delta \phi - F^\mu $$

where $\delta_x \mathcal{L} = \partial_\mu F^\mu$, and $\delta_\phi \mathcal{L}
= \partial_\mu (\partial_{\partial_\mu \phi} \mathcal{L} \delta \phi) $, and so
we see that our partial variation of $\mathcal{L}$ with respect to only $x$ is

$$ \delta \partial_x \mathcal{L} = -\frac{\partial}{\partial x_\mu} j^\mu $$

(no summation convention - this is a vector here), and importantly

$$ \partial_\mu j^\mu = v_M \cdot \delta \partial_x \mathcal{L} = 0 $$

as required since $\partial_\mu j^\mu = \delta \mathcal{L} - \delta \mathcal{L}
= 0$.

That is Noether's theorem. I do not fully understand it, and to me it seems
weird that the ``generators'' of these variations (the terms like $F^\mu$ inside
the derivative) are not the same. But anyhow, Noether's theorem states that
$j^\mu$ is conserved, so $\partial_\mu j^\mu = 0$ (which means the Minkowski
metric stays the same in the way we described). Aside from that we will remark
that we assume in general that for any symmetry $\partial_x \mathcal{L} =
\partial_\mu F^\mu$ for some $F^\mu$,

Now where does a conserved quantity arise from. Here we note that if $Q = \int
dx^3 j^0$ then it satisfies

\begin{align*}
  \frac{d}{dt} Q
  &= \int dx^3 \partial_t j^0 \\
  &= - \int dx^3 \nabla \cdot J \\
  &= \int_S dS \cdot J \\
  &= 0
\end{align*}

and os is conserved. One can check that in the case of electromagnetism, $Q$
corresponds precisely to electromagnetic charge.

[End of Noether's theorem rewrite]

Today we will look at certain examples of Noether's theorem. In particular, we
will define the energy-momentum constant as the Noether current arising from
translation invariance. Under the symmetry $x^\mu \mapsto x^\mu + \epsilon^\mu$,
we get $\phi \mapsto \phi(x - \epsilon) \approx \phi(x) - \epsilon^\mu
\partial_\mu \phi + \dots$. We also get $\delta \mathcal{L} = -\epsilon^\mu
\partial_\mu \mathcal{L}$. Consequently we get energy-momentum tensor

$$ T^\mu_\nu = j^\mu_\nu = \partial_{\partial_\mu \phi} \mathcal{L} \partial_\nu
\phi - \delta^\mu_\nu \mathcal{L} $$

As Noether current, this is conserved, meaning $\partial_\mu T^\mu_\nu = 0$. We
consequently get conserved quantities:

$$ E = \int d^3x T^{00} $$
$$ p^i = \int d^3xT^{0i} $$

or equivalently, the 4-momentum is conserved:

$$ p^\nu = \int d^3x T^{0\nu}. $$

[Lecturer computes example in Klein-Gordon case.] In many cases, such as the
Klein-Gordon case we find the energy-momentum tensor to be symmetric, but this
need not be the case in general. However, just as the Lagrangian gives rise to
the same physics if we add a total derivative, one can show that the
energy-momentum tensor has a similar symmetry, and if a particular choice of
total derivative is made, we find that we can always make $T^{\mu \nu}$
symmetric.

More specifically, we find that if we use the formula from general relativity
for the energy-momentum tensor:

$$ T_{\mu\nu}(x) = \frac{-2}{\sqrt{-g}} \partial_{g^{\mu\nu}} (\sqrt{-g}
\tilde{\mathcal{L}}) $$

for $g = \text{det}(g_{\mu \nu})$ then we always get a symmetric result. By
using the Minkowski metric we can recover our original result.

\section{Canonical Quantisation}

In the Advanced Quantum Field Theory course in Lent, we will use the Lagrangian
directly to formulate the path integral version of quantum mechanics. Here,
however, we will stick to the Hamiltonian approach. For that we do a small
review of Hamiltonian mechanics by starting with the Lagrangian

$$ L(q, \dot{q}) = \frac{1}{2}\dot{q}^2 - V(q) $$

We then define the \textbf{momentum conjugate to $q$} to be $p =
\partial_{\dot{q}} L$. Performing the Legendre transform we then can define the
\textbf{Hamiltonian} to be

$$ H = p\dot{q} - \mathcal{L} = \frac{1}{2}p^2 + V(q) = H(p, q) $$

since $H$ is really seen as a functino of $p$ and $q$. It is also clear that
Hamiltonian can be seen as the total energy of a point in state space $(p, q)$.

Now, we generalise our system to $N$ particles as

$$ H = \sum_{i=1}^N p_i q_i - \mathcal{L} $$

and introduce the \textbf{Poisson
  bracket} (which is related to the commutator) as 

$$ \{F, G\} = \sum_{i = 1}^N \partial_{q_i} F \partial_{p_i} G - \partial_{p_i}
F \partial_{q_i} G $$

for functions $F, G$ of $p, q$. We can then get the following important result:
the Hamiltonian is the generator of time evolution, meaning that $\forall F(p,
q)$ we find

$$ \dot{F} = \{H, F\}. $$

An important special case of this is that we can write the principle of least
action by applying the time evolution property to $p$ and $q$ asking

\begin{align*}
  \dot{q}_i &= \{H, q_i\} = \partial_{p_i} H \\
  \dot{p}_i &= -\partial_{q_i} H
\end{align*}

and for any conserved $Q$, we get $\{H, Q\} = 0$. Finally, we note that these
also satisfy the property that

$$ \{q_i, p_j\} = \delta_{ij}$$

[End of lecture 5]

How do translate this Hamiltonian to field theory? Firstly we define the
\textbf{momentum conjugate to the field} as

$$ \pi = \partial_{\partial_{\dot{\phi}}} \mathcal{L} $$
and we apply the Legendre transform to get the \textbf{Hamiltonian density}

$$ \mathcal{H} = \pi \dot{\phi} - \mathcal{L} = \mathcal{H}(\phi, \pi) $$

and, as expected the normal Hamiltonian can then be calculated as

$$ H = \int d^3x \mathcal{H}. $$

As a check, and example, if we use $\mathcal{L} = \frac{1}{2}\partial_\mu \phi
\partial^\mu \phi - V(\phi)$ (a ``suitably'' general Lagrangian) we get

$$ \mathcal{H} = \frac{1}{2} \dot{\phi}^2 + \frac{1}{2} | \nabla \phi|^2 +
V(\phi) = T^{00} $$

the energy density, as one might hope.

\subsection{The Canonical Quantisation}

Now to actually quantise things we convert the Poisson bracket $\{\}$ to the
commutator $\frac{1}{i\hbar} []$, with overall quantities staying the same. That
means that for $\pi, \phi$ we get that

$$ [\pi_i, \pi_j] = [\phi_i, \phi_j] = 0$$

but that

$$ [ \pi_i, \phi_j] = i\hbar \delta(x - y), $$

as one might expect. The time evolution property of $H$ becomes 

$$ H \ket{\psi} = -i \hbar \partial_t \ket{\psi}. $$

In particular that means that for any conserved quantity (commutes with $H$), we
can simultaneously diagonalise these to get a simultaneous basis $H$ and the
conserved quantity. Also, note here that we later set $\hbar = 1$.

Anyways, using the standard Hamiltonian approach we then find that

$$ H = \int d^3x \frac{1}{2} \pi^2 + \frac{1}{2} |\nabla \phi|^2 + V(\phi) $$

In practice, this is very hard to calculate, so not very helpful (one would have
to diagonalise this function to find the energy eigenstates...). Instead we find
some other ways.

Finally, on interpretation, we note that the eigenvalues have become functions
$f : \mathbb{R}^3 \to \mathbb{R}$, which we assume span the Hilbert space as
usual. However, this has never been formalised mathematically, so it is a bit of
a grey area... [End of lecture 6]

\end{document}