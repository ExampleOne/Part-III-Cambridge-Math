\documentclass{article}
\usepackage[utf8]{inputenc}
\usepackage{mathtools}
\usepackage{amssymb}
\usepackage{graphicx}
\usepackage{listings}
\usepackage{float}
\usepackage{gensymb}
\usepackage{amsthm}
\usepackage{longtable}
\usepackage{adjustbox}
\usepackage{physics}

\theoremstyle{definition}
\newtheorem{definition}{Definition}
\newtheorem{proposition}{Proposition}

\title{Symmetries, Fields, and Particles}
\author{quinten tupker}
\date{October 10 2020 - \today}

\begin{document}

\maketitle

\section*{Introduction}

These notes are based on the course lectured by Ben Allanach in Michaelmas 2020.
Due to the measures taken in the UK to limit the spread of
Covid-19, these lectures were delivered online. These are not meant to be an
accurate representation of what was lectures. They solely represent a mix of
what I thought was the most important part of the course, mixed in with many
(many) personal remarks, comments and digressions... Of course, any
corrections/comments are appreciated.

To begin the course, the lecturer reminds of the definition of a group. I will
not repeat this definition here. Groups, and Lie groups in particular, are
essential in particle physics as a means of keeping track of the symmetries of
particles. Here we get two kinds of symmetries:

\begin{itemize}
\item An \textbf{internal symmetry} is an inherent property of the
  fields/particles themselves. For example, one can rotate through quark
  colours, and in fact, we find that in order to make this possible, we require
  the existence of a force carrying particle (a gluon) to be involved. And to
  conserve colours, gluons contain a mix of colours to do so. The Lie group
  structure enforces colour conservation here... Since these colour rotations can be different at
  different points in space and time, these symmetries can also be
  called a \textbf{local symmetry} or a \textbf{gauge symmetry}.
\item A \textbf{global symmetry} is a symmetry that leaves something the same
  across all space and time. 
\item A \textbf{external symmetry} is a symmetry involving spacetime
  coordinates. This includes symmetry under translation and Lorentz
  transformations. From these symmetries we get conserved structures (momentum,
  angular momentum, and energy). The Poincare group contains all these
  symmetries. 
\end{itemize}

In terms of particles, bosons carry forces, and these includes gluons (strong
force), photons (electromagnetic force), $Z^0, W^\pm$ carries the electroweak
force. It has also been hypothesised that the graviton (spin 2) carries gravity,
although it has never been observed. Also, for good symmetries, force carriers
should be massless, but the spontaneous symmetry breaking, through the Higgs
mechanism, can give force carriers mass (such as $W^\pm, Z^0$).

The lecturer provides a standard list of (elementary) fermions. I won't repeat
these.

Just as a note, in the standard model, every particle has a field, and
excitations of this field corresponds to ``instances'' of these particles.
[End of lecture 1]

In lecture 2, the lecturer reviews basic group theory. I won't repeat that here.
[End of lecture 2]

We continue looking at group properties, etc. Here are some definitions.

\begin{definition}
  The \textbf{inner automorphism} associated with $g \in G$ is $\phi_g(h) =
  ghg^{-1}$.
\end{definition}

We remark that treating elements of $G$ as automorphisms in this way we see that
$G / Z(G)$ is always a normal subgroup of $\text{Aut}(G)$.

\begin{definition}
  The \textbf{semi-direct product} of groups $H, G$, written $H \ltimes G$, has the
  product rule $(h, g)(h', g') = (hh', g\phi_h(g'))$, and inverse rule $(h,
  g)^{-1} = (h^{-1}, \phi_{h^{-1}}(g))$.
\end{definition}

here we can see that $H \equiv H \ltimes G / G$, and that $D_n \equiv
\mathbb{Z}_2 \ltimes \mathbb{Z}_n$ as expected.

\begin{definition}
  The \textbf{commutator subgroup} or \textbf{derived subgroup} of group $G$, denoted $[G, G]$,
  the group generated by the commutators of $G$. These are always normal, and a
  group is called \textbf{perfect} when it equals its own commutator subgroup.
\end{definition}

\section{Matrix Groups}

The lecturer describes the general linear group, orthogonal group, special
orthogonal group, unitary group, and special unitary group. He also describes
their dimensionality. In particular, using their properties, he shows that
$O(n)$ has $\frac{1}{2}n(n - 1)$ free parameters, and $SU(n)$ has $n^2 - 1$
parameters. He also mentions the famous fact that topologically $SU(2) \equiv
S^3$.

\end{document}