\documentclass{article}
\usepackage[utf8]{inputenc}
\usepackage{mathtools}
\usepackage{amssymb}
\usepackage{graphicx}
\usepackage{listings}
\usepackage{float}
\usepackage{gensymb}
\usepackage{amsthm}
\usepackage{longtable}
\usepackage{adjustbox}
\usepackage{physics}

\theoremstyle{definition}
\newtheorem{definition}{Definition}
\newtheorem{proposition}{Proposition}

\title{Distribution Theory and Applications}
\author{quinten tupker}
\date{October 9 2020 - \today}

\begin{document}

\maketitle

\section*{Introduction}

These notes are based on the course lectured by Dr Anthony Ashton in Michaelmas
2020. Due to the measures taken in the UK to limit the spread of
Covid-19, these lectures were delivered online. These are not meant to be an
accurate representation of what was lectures. They solely represent a mix of
what I thought was the most important part of the course, mixed in with many
(many) personal remarks, comments and digressions... Of course, any
corrections/comments are appreciated.

Unfortunately, this course does not require any knowledge of measure theory, and
the like, but it does focus a bit more on the applications of the matter.

\section{Motivation}

Let's start with some questions that motivate the area... Ashton starts by
asking, what is the derivative of the Dirac $\delta$ (although really, at this
point we may as well ask what the Dirac $\delta$ is in the first place).

[In the end I decided not to take this course.]

\end{document}