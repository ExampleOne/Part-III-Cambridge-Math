\documentclass{article}
\usepackage[utf8]{inputenc}
\usepackage{mathtools}
\usepackage{amssymb}
\usepackage{graphicx}
\usepackage{listings}
\usepackage{float}
\usepackage{gensymb}
\usepackage{amsthm}
\usepackage{longtable}
\usepackage{adjustbox}
\usepackage{physics}
\usepackage[makeroom]{cancel}

\theoremstyle{definition}
\newtheorem{definition}{Definition}
\newtheorem{proposition}{Proposition}
\newtheorem{example}{Example}

\title{Non-Equilibrium Statistical Field Theory}
\author{quinten tupker}
\date{October 13 2020 - \today}

\begin{document}

\maketitle

\section*{Introduction}

These notes are based on the course lectured by Johannes Pausch in Michaelmas
2020. Due to the measures taken in the UK to limit the spread of
Covid-19, these lectures were delivered online. These are not meant to be an
accurate representation of what was lectures. They solely represent a mix of
what I thought was the most important part of the course, mixed in with many
(many) personal remarks, comments and digressions... Of course, any
corrections/comments are appreciated.

Non-Equilibrium Statistical Field Theory is the study of statistical properties
of field theory that is changing in time (which is what non-equilibrium refers
to here).

\section{Master Equations}

Firstly, we try to derive some master equations to use in our model. How do we
model our system? We first discretise time into steps $t_n$. For some reason, in
this field by convention, time travels from right to left. Anyways, having
discretised time, we model systems as Markov chains with $N(t_n)$ being the
number of particles ``present'' at $t_n$.

We can then write the \textbf{Chapman-Kolmogorov equation} as

$$ \mathbb{P}(N(t_3) | N(t_1)) = \sum_{N(t_2)} \mathbb{P}(N(t_3) | N(t_2))
\mathbb{P}(N(t_2) | N(t_1)) $$

Now we want to take the continuum limit of this equation by defining

$$ W_t(N' | N) = \partial_{t'} \mathbb{P}(N'(t') | N(t)) |_{t' = t} $$

then we can write

$$ \mathbb{P}(N'(t + \Delta t) | N(t)) \approx \mathbb{P}(N'(t) | N(t)) + \Delta
t W_t(N' | N) = \delta_{N'(t), N(t)} + \Delta W_t(N' | N) $$

Notice that since probabilities add to 1, $\sum_{N'} W_t(N' | N) = 0$, so in
particular $W_t{N | N} = - \sum_{N' \neq N} W_t(N' | N)$. Using
that we can rewrite the Chapman-Kolmogorov equation as

\begin{align*}
  \mathbb{P}(N(t_3) | N(t_1)) - \mathbb{P}(N(t_2) | N(t_1))
  &= \sum_{N(t_2)} \mathbb{P}(N(t_3) | N(t_2)) \mathbb{P}(N(t_2) | N(t_1)) \\
  &= \Delta t \sum_{N_2 \neq N_3} W_{t_2}(N_3 | N_2) \mathbb{P}(N_2 | N_1) -
    W_{t_3}(N_2 | N_3) \mathbb{P}(N_3 | N_1)
\end{align*}

Taking limits and removing the $|N_1$ part in the conditional probabilities for
notational simplicity we get the final form of our \textbf{Master Equation}
(which has no other name?)

$$ \partial_t \mathbb{P}(N) = \sum_{N' \neq N} W_t(N | N') \mathbb{P}(N') -
W_t(N' | N) \mathbb{N} $$

where the first term in the sum can be called the \textbf{gain}, and the second
subtracted term can be called the \textbf{loss} (we can interpret them as such).
To build intuition we provide a simple example.

\begin{example}
  We consider a simple extinction process where a group of particles are
  gradually disappearing. Here $W(N - 1|N) = N\epsilon, W(N|N) = -N\epsilon$ for
  some $\epsilon > 0$ while all other values of the matrix $W$ are zero. As such
  we get a simple master equation
  $$ \partial_t \mathbb{P}(N) = \epsilon(N + 1) \mathbb{P}(N + 1) - \epsilon N
  \mathbb{P}(N) $$
  [End of lecture 1]
\end{example}

\end{document}