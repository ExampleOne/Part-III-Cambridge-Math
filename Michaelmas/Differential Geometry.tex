\documentclass{article}
\usepackage[utf8]{inputenc}
\usepackage{mathtools}
\usepackage{amssymb}
\usepackage{graphicx}
\usepackage{listings}
\usepackage{float}
\usepackage{gensymb}
\usepackage{amsthm}
\usepackage{longtable}
\usepackage{adjustbox}
\usepackage{physics}

\theoremstyle{definition}
\newtheorem{definition}{Definition}
\newtheorem{proposition}{Proposition}

\title{Differential Geometry}
\author{quinten tupker}
\date{October 8 2020 - \today}

\begin{document}

\maketitle

\section*{Introduction}

These notes are based on the course lectured by dr Jack E Smith in Michaelmas
2020.  Due to the measures taken in the UK to limit the spread of
Covid-19, these lectures were delivered online. These are not meant to be an
accurate representation of what was lectures. They solely represent a mix of
what I thought was the most important part of the course, mixed in with many
(many) personal remarks, comments and digressions... Of course, any
corrections/comments are appreciated.

Unlike some of the other courses, there is no real introduction here, and we
jump straight into the content!

\section{Manifolds and Smooth Maps}

Manifolds are spaces that locally look like $\mathbb{R}^n$. Formally this is:

\begin{definition}
  $X$ is a \textbf{topological $n$ manifold} if $X$ is a second countable
  Hausdorff topological space such that $\forall p \in X \exists \text{open} U
  \ni p$ and open $V \subseteq \mathbb{R}^n$ and homeomorphism $\phi : U \to V$.
\end{definition}

Here,

\begin{definition}
  Topological space $X$ is \textbf{Hausdorff} if for every distinct $x, y \in X$ there
  exists open $U \ni x, V \ni y$ in $X$ such that $U \cap V = \emptyset$.
\end{definition}

and

\begin{definition}
  Topological space $X$ is \textbf{second countable} if there exists a set of
  open sets $\mathcal{U}$ st that every open set in $X$ can be written as a
  union of sets in $\mathcal{U}$
\end{definition}

Since these two properties transfer to subsets, any subset of a topological $n$
manifold is also a topological $n$ manifold. Also, to give some more intuition,
the condition that $X$ is a seond countable Hausdorff topological space is
exactly equivalent to the condition that $X$ is metrizable and has countably
many components. It is just tradition that it is defined as above. Some more
defintions. Above, $\phi$ is called the \textbf{chart}, $U$ is called the
\textbf{coordinate patch}, although in some cases can also be called the chart.
The functions $x_1 \circ \phi, \dots, x_n \circ \phi$ (so the components of
the result) are called the \textbf{local coordinates}, and $\phi^{-1}$ is called
the \textbf{paramaterisation}, although that term is not used that frequently.
Finally, for overlapping charts, we can define the \textbf{transition map}
between them as $\phi_2 \circ \phi_1^{-1} : \phi_1 (U_1 \cap U_2) \to \phi_2(U_1
\cap U_2)$.

Now, we want to generalise calculus to manifolds, so it makes sense to start by
trying to generalise the notion of smoothness. The simplest approach would be to
say that $f$ is smooth on $X$ if it is smooth on the local coordinates. The
issue then arises that this may not be consistent with smoothness on other
charts (where these overlap). As such, we need to require that the transition
maps are smooth as well. Consequently we do the following:

\begin{definition}
  The \textbf{atlas} of a manifold is a collection of charts of a topological
  n manifold that covers all of $X$.
\end{definition}

An atlas is \textbf{smooth} if all transition maps are smooth, and a map $f$ is
\textbf{smooth} on atlas $\mathfrak{A}$ if $f \circ \phi_\alpha^{-1}$ is smooth
$\forall \alpha$. As a result all local coordinate functions are smooth. Now,
really, specifying the atlas precisely all the time is somewhat tedious, and
somehow not the point, so we want a degree of flexibility. For this we define

\begin{definition}
  Two atlases are \textbf{smoothly equivalent} if their union is smooth.
\end{definition}

Note that this forms an equivalence relation (apply the chain rule on transition
functions for transitivity).

\begin{definition}
  A \textbf{smooth structure} is an equivalence class of atlases.
\end{definition}

As we hope, we do indeed have that if a function is smooth wrt to an atlas, it
is smooth wrt to any atlas in its smooth structure. Also, we can define a
maximal atlas to be the union of all the atlases in a smooth structure, if we
deem that to be convenient (includes trivial changes like translating or scaling
the local coordinates).

\begin{definition}
  A \textbf{smooth $n$ manifold} is a topological $n$ manifold with a smooth
  structure.
\end{definition}

Note that under the product topology, the product of two manifolds naturally
forms a new (smooth) manifold. Also, a remarkable fact is that for $n=1, 2, 3$
all topological $n$ manifolds have an essentially unique smooth structure,
whereas this breaks down for $n \geq 4$. Also, a new chart is said to be
\textbf{compatible} with an atlas if when added to the atlas, the atlas remains
smooth.

Finally, to give a concrete example of a manifold, we may consider $S^n$, which
forms a manifold with two charts: one being the sphere without the North pole,
and the other being the sphere without the South pole, $U_\pm$ with charts

$$ \phi_\pm(y_0, \dots, y_n) = \frac{1}{1 \mp y_0} (y_1, \dots, y_n), $$

where the local coordinates are referred to as $x^\pm$. [End of DG1]

\section{Forming Manifolds from Sets (Instead of Topological Spaces)}

We observe that an atlas can generate a topology. In particular if we consider
the data

\begin{itemize}
\item set $X$
\item subsets $U_\alpha \subseteq X$
\item open sets $V_\alpha \subseteq \mathbb{R}^n$
\item bijections $\phi_\alpha : U_\alpha \to V_\alpha$ that have smooth
  transition functions, and $\forall \alpha, \beta, \phi_\alpha(U_\alpha \cap
  U_\beta)$ is open in $V_\alpha$ (weird but useful)
\end{itemize}

then we see that if we declare $U$ to be open iff $\phi_\alpha(U \cap U_\alpha)$
is open $\forall \alpha$, then this forms a topology (easy) and 

\begin{proposition}
  Apart from the possible failure of Hausdorff and second countable, using the
  above data as specified turns $X$ into a topological $n$ manifold, and
  $\{\phi_\alpha\}$ into a smooth atlas (so we have a smooth manifold).
\end{proposition}

\begin{proof}
  It suffices to show that $U_\alpha$ are open and that $\phi_\alpha$ are
  homeomorphisms (smoothness follows from the smoothness of the transition
  functions). As such it is sufficient to show that some $U \subseteq U_\alpha$
  is open iff $\phi_\alpha(U)$ is open in $V_\alpha$ (this is to show that
  $\phi_\alpha$ is a homeomorphism, which implies taht $U_\alpha$ is open by the
  openness of $V_\alpha$). One direction is clear: if $U$ is open, then by
  declaration, $\phi(U \cap U_\alpha = U)$ is open. Conversely, if $\phi(U)$ is
  open then we want that $\forall \beta, \phi_\beta(U \cap U_\beta)$ is open,
  but we observe that
  \begin{align*}
    \phi_\beta(U \cap U_\beta)
    &= \phi_\beta \circ \phi_\alpha^{-1} (\phi_\alpha(U \cap U_\beta)) \\
    &= (\phi_\alpha \circ \phi_\beta^{-1})^{-1}
      (\phi_\alpha(U) \cap \phi_\alpha(U_\alpha \cap U_\beta))
  \end{align*}
  Here $\phi_\alpha(U_\alpha \cap U_\beta)$ is open as an intersection of
  $V_\alpha \cap V_\beta$, and $\phi_\alpha(W)$ is open by assumption. The
  transition function is continuous by assumption. So done. 
\end{proof}

Finally, we note that we can define this set of $\phi$s and $U$s and $V$s, etc.
to be a ``pseudo-chart'', and can define a ``smooth pseudo-structure,'' etc.
from here. All we really want to show is that the toplogy is secondary once we
have a good set of functions. In fact, using this approach we can skip $X$
entirely, and start from sets $U_\alpha$ that we identify with different real
spaces, and then stitch together with arbitrary pseudo-charts. That can
definitely be done, but generally is quite complicated with many more moving
parts, and so we usually, at least for the purpose of a course, start with a
structure in mind, and turn that into a manifold, instead of stitching an
arbitrary one together (although that may be a good source of counter examples).

Unfortunately, it does remain the case that showing Hausdorff and
second-countable is still hard, although there are some tricks to do so. For
second countability, using the subset of all rational balls will always work if
the number of charts is countable. For Hausdorffness, as long as two points live
in the same chart, we are immediately done, so combining charts and considering
the exceptions can be an efficient approach.

\end{document}